\documentclass[12pt]{article}
\usepackage[table]{xcolor}
\usepackage{fancyhdr}
\usepackage{array}
\usepackage{longtable}
\usepackage{booktabs}
\usepackage{amsmath}
\usepackage{multicol}
\usepackage{siunitx}
\usepackage{graphicx}
\usepackage{setspace}
\usepackage{xcolor}
\usepackage{enumitem}
\usepackage{caption}
\doublespacing
\singlespacing
\usepackage{amssymb}
\usepackage[a4paper, top=1in, bottom=1in, left=1.25in, right=1in]{geometry}
\usepackage{times}
\onehalfspacing  
\pagestyle{fancy}
\fancyhf{}
\renewcommand{\footrulewidth}{0.4pt}
\fancyhead[L]{\textbf{GATE 2019 General Aptitude(GA) Set-6}}
\fancyfoot[L]{GA}
\fancyfoot[R]{\thepage /3}
\setlength{\headheight}{15pt}
\setlength{\headsep}{20pt}
\begin{document}
\noindent\textbf{Q.1 - Q.5 carry one mark each.}
\begin{enumerate}[label=Q.\arabic*]
	\item The expenditure on the project \underline{\hspace{2cm}}as follows: equipment Rs.20 lakhs, salaries Rs.12 lakhs, and contingency Rs.3 lakhs.
		\begin{multicols}{2}
			\begin{enumerate}[label=(\Alph*)]
				\item break down
				\item break
				\item breaks down
				\item breaks
			\end{enumerate}
		\end{multicols}

	\item The search engine's business model \underline{\hspace{2cm}} around the fulcrum of trust.
		\begin{multicols}{2}
			\begin{enumerate}[label=(\Alph*)]
				\item revolves
				\item plays
				\item sinks
				\item bursts
			\end{enumerate}
		\end{multicols}

	\item Two cars start at the same time from the same location and go in the same direction. The speed of the first car is 50 km/h and the speed of the second car is 60 km/h. The number of hours it takes for the distance between the two cars to be 20 km is \underline{\hspace{2cm}}.
		\begin{multicols}{2}
			\begin{enumerate}[label=(\Alph*)]
				\item 1
				\item 2
				\item 3
				\item 6
			\end{enumerate}
		\end{multicols}

	\item Ten friends planned to share equally the cost of buying a gift for their teacher. When two of them decided not to contribute, each of the other friends had to pay Rs. 150 more. The cost of the gift was Rs. \underline{\hspace{2cm}}.
		\begin{multicols}{2}
			\begin{enumerate}[label=(\Alph*)]
				\item 666
				\item 3000
				\item 6000
				\item 12000
			\end{enumerate}
		\end{multicols}

	\item A court is to a judge as \underline{\hspace{2cm}} is to a teacher.
		\begin{multicols}{2}
			\begin{enumerate}[label=(\Alph*)]
				\item a student
				\item a punishment
				\item a syllabus
				\item a school
			\end{enumerate}
		\end{multicols}
\end{enumerate}

\textbf{Q.6 to Q.10 carry two marks each.}

\begin{enumerate}[label=\textbf{Q.\arabic*}., start=6, leftmargin=*]
	\item The police arrested four criminals  P, Q, R, and S. The criminals knew each other. They made the following statements:\\
		\quad P says "Q committed the crime"\\
		\quad Q says "S committed the crime"\\
		\quad R says "I did not do it."\\
		\quad S says "What Q said about me is false.

		Assume only one of the arrested four committed the crime and only one of the statements made above is true. Who committed the crime?
		\begin{multicols}{2}
			\begin{enumerate}[label=(\Alph*)]
				\item P
				\item R
				\item S
				\item Q
			\end{enumerate}
		\end{multicols}

	\item In the given diagram, teachers are represented in the triangle, researchers in the circle and administrators in the rectangle. Out of the total number of the people, the percentage of administrators shall be in the range of \_\_\_.

		\begin{center}
			{
				\includegraphics[width=0.5\linewidth]{ch_ga_q7.jpg}
			}
		\end{center}

		\begin{multicols}{2}
			\begin{enumerate}[label=(\Alph*)]
				\item 0 to 15
				\item 16 to 30
				\item 31 to 45
				\item 46 to 60
			\end{enumerate}
		\end{multicols}

	\item "A recent High Court judgement has sought to dispel the idea of begging as a disease which leads to its stigmatization and criminalization and to regard it as a symptom. The underlying disease is the failure of the state to protect citizens who fall through the social security net."

		Which one of the following statements can be inferred from the given passage?
		\begin{enumerate}[label=(\Alph*)]
			\item Beggars are lazy people who beg because they are unwilling to work
			\item Beggars are created because of the lack of social welfare schemes
			\item Begging is an offence that has to be dealt with firmly
			\item Begging has to be banned because it adversely affects the welfare of the state
		\end{enumerate}
	\item In a college, there are three student clubs. Sixty students are only in the Drama club, 80 students are only in the Dance club, 30 students are only in the Maths club, 40 students are in both Drama and Dance clubs, 12 students are in both Dance and Maths clubs, 7 students are in both Drama and Maths clubs, and 22 students are in all the clubs. If 75\% of the students in the college are not in any of these clubs, then the total number of students in the college is \_\_\_.
		\begin{multicols}{2}
			\begin{enumerate}[label=(\Alph*)]
				\item 1000
				\item 975
				\item 900
				\item 225
			\end{enumerate}
		\end{multicols}

	\item Three of the five students allocated to a hostel put in special requests to the warden. Given the floor plan of the vacant rooms, select the allocation plan that will accommodate all their requests.

		Request by X: Due to pollen allergy, I want to avoid a wing next to the garden.\\
		Request by Y: I want to live as far from the washrooms as possible, since I am very sensitive to smell.\\
		Request by Z: I believe in Vaastu and so want to stay in the South-west wing.

		The shaded rooms are already occupied. WR is washroom.

		\begin{multicols}{2}
			\begin{enumerate}[label=(\Alph*)]
				\item \includegraphics[width=0.8\linewidth]{ch_ga_q10a.jpg}
				\item \includegraphics[width=0.8\linewidth]{ch_ga_q10b.jpg}
				\item \includegraphics[width=0.8\linewidth]{ch_ga_q10c.jpg}
				\item \includegraphics[width=0.8\linewidth]{ch_ga_q10d.jpg}
			\end{enumerate}
		\end{multicols}
		\vspace{\baselineskip}
		\begin{center}
			{\textbf{END OF THE QUESTION PAPER}}
		\end{center}
		\newpage
\end{enumerate}
\fancyhead[L]{GATE 2019}
\fancyhead[R]{Chemical engineering}
\fancyfoot[L]{CH}
\setcounter{page}{1}
\fancyfoot[R]{\thepage/14}
\noindent\textbf{Q.1 - Q.25 carry one mark each.}
\begin{enumerate}[label=Q.\arabic*]

	\item A system of $n$ homogeneous linear equations containing $n$ unknowns will have non-trivial solutions if and only if the determinant of the coefficient matrix is
		\begin{multicols}{2}
			\begin{enumerate}[label=(\Alph*)]
				\item 1
				\item $-1$
				\item 0
				\item $\infty$
			\end{enumerate}
		\end{multicols}

	\item The value of the expression $\lim\limits_{x \to \frac{\pi}{2}} \frac{\tan x}{x}$ is
		\begin{multicols}{2}
			\begin{enumerate}[label=(\Alph*)]
				\item $\infty$
				\item 0
				\item 1
				\item $-1$
			\end{enumerate}
		\end{multicols}

	\item Consider a rigid, perfectly insulated, container partitioned into two unequal parts by a thin membrane (see figure). One part contains one mole of an ideal gas at pressure $P_1$ and temperature $T_1$ while the other part is evacuated. The membrane ruptures, the gas fills the entire volume and the equilibrium pressure is $P_2 = \frac{P_1}{4}$. If $C_p$ (molar specific heat capacity at constant pressure), $C_v$ (molar specific heat capacity at constant volume) and $R$ (universal gas constant) have the same units as molar entropy, the change in molar entropy ($S_f - S_i$) is
		\begin{center}
			{
				\includegraphics[width=0.5\linewidth]{ch_q3.jpg}
			}
		\end{center}
		\begin{multicols}{2}
			\begin{enumerate}[label=(\Alph*)]
				\item $C_p \ln 2 + R \ln 4$
				\item $-C_p \ln 2 + R \ln 4$
				\item $R \ln 4$
				\item $C_p \ln 2$
			\end{enumerate}
		\end{multicols}

	\item For a single component system, vapor (subscript $g$) and liquid (subscript $f$) coexist in mechanical, thermal and phase equilibrium when
		\begin{multicols}{2}
			\begin{enumerate}[label=(\Alph*)]
				\item $u_g = u_f$ (equality of specific internal energy)
				\item $h_g = h_f$ (equality of specific enthalpy)
				\item $s_g = s_f$ (equality of specific entropy)
				\item $g_g = g_f$ (equality of specific Gibbs free energy)
			\end{enumerate}
		\end{multicols}

	\item For a binary nonideal A-B mixture exhibiting a minimum boiling azeotrope, the activity coefficients $\gamma_i$ ($i = A, B$), must satisfy
		\begin{multicols}{2}
			\begin{enumerate}[label=(\Alph*)]
				\item $\gamma_A > 1, \gamma_B > 1$
				\item $\gamma_A < 1, \gamma_B > 1$
				\item $\gamma_A = 1, \gamma_B = 1$
				\item $\gamma_A < 1, \gamma_B < 1$
			\end{enumerate}
		\end{multicols}

	\item For a fully-developed turbulent hydrodynamic boundary layer for flow past a flat plate, the thickness of the boundary layer increases with distance $x$ from the leading edge of the plate, along the free-stream flow direction, as
		\begin{multicols}{2}
			\begin{enumerate}[label=(\Alph*)]
				\item $x^{0.5}$
				\item $x^{1.5}$
				\item $x^{0.4}$
				\item $x^{0.8}$
			\end{enumerate}
		\end{multicols}

	\item Consider a cylinder (diameter $D$ and length $D$), a sphere (diameter $D$) and a cube (side length $D$). Which of the following statements concerning the sphericity ($\Phi$) of the above objects is true:
		\begin{multicols}{2}
			\begin{enumerate}[label=(\Alph*)]
				\item $\Phi_{\text{sphere}} > \Phi_{\text{cylinder}} > \Phi_{\text{cube}}$
				\item $\Phi_{\text{sphere}} = \Phi_{\text{cylinder}} = \Phi_{\text{cube}}$
				\item $\Phi_{\text{sphere}} < \Phi_{\text{cylinder}} < \Phi_{\text{cube}}$
				\item $\Phi_{\text{cylinder}} > \Phi_{\text{sphere}} = \Phi_{\text{cube}}$
			\end{enumerate}
		\end{multicols}

	\item Prandtl number signifies the ratio of
		\begin{multicols}{2}
			\begin{enumerate}[label=(\Alph*)]
				\item Momentum Diffusivity / Thermal Diffusivity
				\item Mass Diffusivity / Thermal Diffusivity
				\item Thermal Diffusivity / Momentum Diffusivity
				\item Thermal Diffusivity / Mass Diffusivity
			\end{enumerate}
		\end{multicols}

	\item Pool boiling equipment operating above ambient temperature is usually designed to operate
		\begin{multicols}{2}
			\begin{enumerate}[label=(\Alph*)]
				\item far above the critical heat flux
				\item near the critical heat flux
				\item far above the Leidenfrost point
				\item near the Leidenfrost point
			\end{enumerate}
		\end{multicols}

	\item The desired liquid-phase reaction
		\[ D + E \xrightarrow{k_1} F \quad \text{with rate } r_F = k_1 C_D^2 C_E^{0.3} \]
		is accompanied by an undesired side reaction
		\[ D + E \xrightarrow{k_2} G \quad \text{with rate } r_G = k_2 C_D^{0.4} C_E^{1.5} \]
		Four isothermal reactor schemes (CSTR: ideal Continuous-Stirred Tank Reactor, PFR: ideal Plug Flow Reactor) for processing equal molar feed rates of D and E are shown in figure. Each scheme is designed for the same conversion. The scheme that gives the most favorable product distribution is:

		\begin{multicols}{2}
			\begin{enumerate}[label=(\Alph*)]
				\item \includegraphics[width=0.8\linewidth]{ch_q10a.jpg}
				\item \includegraphics[width=0.8\linewidth]{ch_q10b.jpg}
				\item \includegraphics[width=0.8\linewidth]{ch_q10c.jpg}
				\item \includegraphics[width=0.8\linewidth]{ch_q10d.jpg}
			\end{enumerate}
		\end{multicols}

	\item For a first order reaction in a porous spherical catalyst pellet, diffusional effects are most likely to lower the observed rate of reaction for
		\begin{multicols}{2}
			\begin{enumerate}[label=(\Alph*)]
				\item slow reaction in a pellet of small diameter
				\item slow reaction in a pellet of large diameter
				\item fast reaction in a pellet of small diameter
				\item fast reaction in a pellet of large diameter
			\end{enumerate}
		\end{multicols}

	\item A thermocouple senses temperature based on the
		\begin{multicols}{2}
			\begin{enumerate}[label=(\Alph*)]
				\item Nernst Effect
				\item Maxwell Effect
				\item Seebeck Effect
				\item Peltier Effect
			\end{enumerate}
		\end{multicols}

	\item The correct expression for the Colburn $j$-factor for mass transfer that relates Sherwood number (Sh), Reynolds number (Re) and Schmidt number (Sc) is
		\begin{multicols}{4}
			\begin{enumerate}[label=(\Alph*)]
				\item $\frac{\text{Sh}}{(\text{Re})(\text{Sc})}$
				\item $\frac{\text{Sh}}{(\text{Re})^{0.5}(\text{Sc})}$
				\item $\frac{\text{Sh}}{(\text{Re})^{0.5}(\text{Sc})^{0.33}}$
				\item $\frac{\text{Sh}}{(\text{Re})(\text{Sc})^{0.33}}$
			\end{enumerate}
		\end{multicols}

	\item In the drying of non-dissolving solids at constant drying conditions, the internal movement of moisture in the solid has a dominant effect on the drying rate during
		\begin{multicols}{2}
			\begin{enumerate}[label=(\Alph*)]
				\item the initial adjustment period only
				\item the constant rate period only
				\item the falling rate period only
				\item both the initial adjustment and constant rate periods
			\end{enumerate}
		\end{multicols}

	\item Three distillation schemes for separating an equimolar, constant relative volatility ABC mixture into nearly pure components are shown. The usual simplifying assumptions such as constant molal overflow, negligible heat loss, ideal trays are valid. All the schemes are designed for minimum total reboiler duty. Given that the relative volatilities are in the ratio $\alpha_2 : \alpha_1 : \alpha_3 = 8:2:1$, the correct option that arranges the optimally-designed schemes in ascending order of total reboiler duty is

		\begin{center}{
				\includegraphics[width=0.5\linewidth]{ch_q15.jpg}
		}\end{center}

		\begin{multicols}{2}
			\begin{enumerate}[label=(\Alph*)]
				\item I, II, III
				\item III, I, II
				\item II, I, III
				\item III, II, I
			\end{enumerate}
		\end{multicols}

	\item Consider the two countercurrent heat exchanger designs for heating a cold stream from $t_{in}$ to $t_{out}$, as shown in figure. The hot process stream is available at $T_{in}$. The inlet stream conditions and overall heat transfer coefficients are identical in both the designs. The heat transfer area in Design I and Design II are respectively $A_{HX1}$ and $A_{HX2}$.

		\begin{center}{
				\includegraphics[width=0.5\linewidth]{ch_q16.jpg}
		}\end{center}

		If heat losses are neglected, and if both the designs are feasible, which of the following statements holds true:

		\begin{multicols}{2}
			\begin{enumerate}[label=(\Alph*)]
				\item $A_{HX2} > A_{HX1}$ \quad $T_{out} < T_{out}''$
				\item $A_{HX2} = A_{HX1}$ \quad $T_{out} = T_{out}''$
				\item $A_{HX2} < A_{HX1}$ \quad $T_{out} > T_{out}''$
				\item $A_{HX2} < A_{HX1}$ \quad $T_{out} = T_{out}''$
			\end{enumerate}
		\end{multicols}

	\item Producer gas is obtained by
		\begin{multicols}{2}
			\begin{enumerate}[label=(\Alph*)]
				\item passing air through red hot coke
				\item thermal cracking of naphtha
				\item passing steam through red hot coke
				\item passing air and steam through red hot coke
			\end{enumerate}
		\end{multicols}

	\item In Kraft process, the essential chemical reagents used in the digester are
		\begin{multicols}{2}
			\begin{enumerate}[label=(\Alph*)]
				\item caustic soda, mercaptans and ethylene oxide
				\item caustic soda, sodium sulphide and soda ash
				\item quick lime, salt cake and dimethyl sulphide
				\item baking soda, sodium sulphide and mercaptans
			\end{enumerate}
		\end{multicols}

	\item The most common catalyst used for oxidation of o-xylene to phthalic anhydride is
		\begin{multicols}{2}
			\begin{enumerate}[label=(\Alph*)]
				\item $\text{V}_2\text{O}_5$
				\item Pd
				\item Pt
				\item Ag
			\end{enumerate}
		\end{multicols}

	\item In petroleum refining operations, the process used for converting paraffins and naphthenes to aromatics is
		\begin{multicols}{2}
			\begin{enumerate}[label=(\Alph*)]
				\item alkylation
				\item catalytic reforming
				\item hydrocracking
				\item isomerization
			\end{enumerate}
		\end{multicols}

	\item The combination that correctly matches the polymer in Group-1 with the polymerization reaction type in Group-2 is \\

		\textbf{Group-1} \hspace{3cm} \textbf{Group-2} \\
		P) Nylon 6 \hspace{3.2cm} I) Condensation polymerization \\
		Q) Polypropylene \hspace{1.95cm} II) Ring opening polymerization \\
		R) Polyester \hspace{2.8cm} III) Addition polymerization

		\begin{multicols}{2}
			\begin{enumerate}[label=(\Alph*)]
				\item P-II, Q-I, R-III
				\item P-I, Q-III, R-II
				\item P-III, Q-I, R-I
				\item P-II, Q-III, R-I
			\end{enumerate}
		\end{multicols}

	\item The product of the eigenvalues of the matrix $\begin{pmatrix} 2 & 3 \\ 0 & 7 \end{pmatrix}$ is \rule{3cm}{0.15mm} (rounded off to one decimal place).

		\item For a hydraulic lift with dimensions shown in figure, assuming $g = 10\,\text{m/s}^2$, the maximum diameter $D_{\text{left}}$ (in m) that lifts a vehicle of mass 1000 kg using a force of 100 N is \rule{3cm}{0.15mm} (rounded off to two decimal places).

			\begin{center}{
					\includegraphics[width=0.5\linewidth]{ch_q23.jpg}
			}\end{center}

		\item The liquid flow rate through an equal percentage control valve, when fully open, is 150 gal/min and the corresponding pressure drop is 50 psi. If the specific gravity of the liquid is 0.8, then the valve coefficient, $C_v$, in gal/(min psi$^{0.5}$) is \rule{3cm}{0.15mm} (rounded off to two decimal places).

		\item Consider a sealed rigid bottle containing CO$_2$ and H$_2$O at 10 bar and ambient temperature. Assume that the gas phase in the bottle is pure CO$_2$ and follows the ideal gas law. The liquid phase in the bottle contains CO$_2$ dissolved in H$_2$O and is an ideal solution. The Henry’s constant at the system pressure and temperature is $H_{CO_2} = 1000$ bar. The equilibrium mole fraction of CO$_2$ dissolved in H$_2$O is \rule{3cm}{0.15mm} (rounded off to three decimal places).

		\item The solution of the ordinary differential equation
			\[ \frac{dy}{dx} + 3y = 1, \text{ subject to the initial condition } y = 1 \text{ at } x = 0, \]
			is
			\begin{multicols}{2}
				\begin{enumerate}[label=(\Alph*)]
					\item $\frac{1}{3}(1 + 2e^{-3x})$
					\item $\frac{1}{3}(5 - 2e^{-3x})$
					\item $\frac{1}{3}(5 + 2e^{-3x})$
					\item $\frac{1}{3}(1 + 2e^{3x})$
				\end{enumerate}
			\end{multicols}

		\item The value of the complex number $i^{-1/2}$ (where $i = \sqrt{-1}$) is
			\begin{multicols}{2}
				\begin{enumerate}[label=(\Alph*)]
					\item $\frac{1}{\sqrt{2}}(1 - i)$
					\item $\frac{1}{\sqrt{2}}i$
					\item $\frac{1}{\sqrt{2}}(-i)$
					\item $\frac{1}{\sqrt{2}}(1 + i)$
				\end{enumerate}
			\end{multicols}

		\item An incompressible Newtonian fluid flows in a pipe of diameter $D_1$ at volumetric flow rate $Q$. Fluid with same properties flows in another pipe of diameter $D_2 = D_1/2$ at the same flow rate $Q$. The transition length required for achieving fully-developed flow is $L_1$ for the tube of diameter $D_1$, while it is $L_2$ for the tube of diameter $D_2$. Assuming steady laminar flow in both cases, the ratio $L_2/L_1$ is:
			\begin{multicols}{2}
				\begin{enumerate}[label=(\Alph*)]
					\item 1/4
					\item 1
					\item 2
					\item 4
				\end{enumerate}
			\end{multicols}

		\item A disk turbine is used to stir a liquid in a baffled tank. To design the agitator, experiments are performed in a lab-scale model with a turbine diameter of 0.05 m and a turbine impeller speed of 600 rpm. The liquid viscosity is 0.001 Pa·s while the liquid density is 1000 kg/m$^3$. The actual application has a turbine diameter of 0.5 m, an impeller speed of 600 rpm, a liquid viscosity of 0.01 Pa·s and a liquid density of 1000 kg/m$^3$. The effect of gravity is negligible. If the power required in the lab-scale model is $P_1$ and the estimated power for the actual application is $P_2$, then the ratio $P_2 / P_1$ is
			\begin{multicols}{2}
				\begin{enumerate}[label=(\Alph*)]
					\item $10^3$
					\item $10^4$
					\item $10^5$
					\item $10^6$
				\end{enumerate}
			\end{multicols}

		\item Consider two non-interacting tanks-in-series as shown in figure. Water enters TANK 1 at $q$ cm$^3$/s and drains down to TANK 2 by gravity at a rate $k \sqrt{h_1}$ (cm$^3$/s). Similarly, water drains from TANK 2 by gravity at a rate of $k \sqrt{h_2}$ (cm$^3$/s) where $h_1$ and $h_2$ represent levels of TANK 1 and TANK 2, respectively (see figure). Drain valve constant $k = 4$ cm$^{2.5}$/s and cross-sectional areas of the two tanks are $A_1 = A_2 = 28$ cm$^2$.

			At steady state operation, the water inlet flow rate is $q_s = 16$ cm$^3$/s. The transfer function relating the deviation variables $\hat{h}_2$ (cm) to flow rate $\hat{q}$ (cm$^3$/s) is

			\begin{center}{
					\includegraphics[width=0.5\linewidth]{ch_q30.jpg}
			}\end{center}

			\begin{multicols}{2}
				\begin{enumerate}[label=(\Alph*)]
					\item $\frac{2}{(56s + 1)^2}$
					\item $\frac{2}{(62s + 1)^2}$
					\item $\frac{2}{(36s + 1)^2}$
					\item $\frac{2}{(49s + 1)^2}$
				\end{enumerate}
			\end{multicols}


		\item Choose the option that correctly matches the step response curves on the left with the appropriate transfer functions on the right. The step input change occurs at time $t = 0$.

			\begin{center}{
					\includegraphics[width=0.5\linewidth]{ch_q31.jpg}
			}\end{center}

			\begin{multicols}{2}
				\begin{enumerate}[label=(\Alph*)]
					\item P - III, Q - IV, R - II, S - I
					\item P - III, Q - I, R - IV, S - II
					\item P - IV, Q - III, R - II, S - I
					\item P - III, Q - II, R - IV, S - I
				\end{enumerate}
			\end{multicols}

		\item 100 kg of a feed containing 50 wt.\% of a solute C is contacted with 80 kg of a solvent containing 0.05 wt.\% of C in a mixer-settler unit. From this operation, the resultant extract and raffinate phases contain 40 wt.\% and 20 wt.\% of C, respectively. If E and R denote the mass of the extract and raffinate phases, respectively, the ratio E/R is
			\begin{multicols}{2}
				\begin{enumerate}[label=(\Alph*)]
					\item 1/4
					\item 1/2
					\item 2/3
					\item 1
				\end{enumerate}
			\end{multicols}

		\item The combination that correctly matches the process in Group-1 with the entries in Group-2 is:

			\begin{center}
				\textbf{Group-1} \hspace{3cm} \textbf{Group-2} \\
				P) Sulfite process \hspace{2.3cm} I) Sulfur mining \\
				Q) Sulfate process \hspace{2.3cm} II) Pulp production \\
				R) Solvay process \hspace{2.3cm} III) Soda ash production \\
				S) Frasch process \hspace{2.3cm} IV) Polyamide production
			\end{center}

			\begin{multicols}{2}
				\begin{enumerate}[label=(\Alph*)]
					\item P-II, Q-IV, R-III, S-I
					\item P-III, Q-IV, R-II, S-I
					\item P-IV, Q-I, R-II, S-III
					\item P-II, Q-I, R-IV, S-III
				\end{enumerate}
			\end{multicols}

		\item If $x$, $y$ and $z$ are directions in a Cartesian coordinate system and $\vec{i}$, $\vec{j}$ and $\vec{k}$ are the respective unit vectors, the directional derivative of the function $w(x,y,z) = x^2 - 3yz$ at the point $(2,0,-4)$ in the direction $(\vec{i} + \vec{j} + 2\vec{k})/\sqrt{6}$ is \rule{3cm}{0.15mm} (rounded off to two decimal places).

		\item Two unbiased dice are thrown. Each dice can show any number between 1 and 6. The probability that the sum of the outcomes of the two dice is divisible by 4 is \rule{3cm}{0.15mm} (rounded off to two decimal places).

		\item The Newton-Raphson method is used to determine the root of the equation $f(x) = e^x - x$. If the initial guess for the root is 0, the estimate of the root after two iterations is \rule{3cm}{0.15mm} (rounded off to three decimal places).

		\item Carbon monoxide (CO) reacts with hydrogen sulphide (H$_2$S) at a constant temperature of 800 K and a constant pressure of 2 bar as:
			\[\text{CO} + \text{H}_2\text{S} \rightleftharpoons \text{COS} + \text{H}_2\]
			The Gibbs free energy of the reaction $\Delta g_r^0 = -22972.3$ J/mol and universal gas constant $R = 8.314$ J/(mol·K). Both the reactants and products can be treated to be ideal gases. If initially only 4 mol of H$_2$S and 1 mol of CO are present, the extent of the reaction (in mol) at equilibrium is \rule{3cm}{0.15mm} (rounded off to two decimal places).

		\item For a given binary system at constant temperature and pressure, the molar volume (in m$^3$/mol) is given by: $v = 30x_A + 20x_B + x_A x_B (15x_A - 7x_B)$, where $x_A$ and $x_B$ are the mole fractions of components A and B, respectively. The volume change of mixing $\Delta v_{mix}$ (in m$^3$/mol) at $x_A = 0.5$ is \rule{3cm}{0.15mm} (rounded off to one decimal place).

		\item Consider a vessel containing steam at 180~$^\circ$C. The initial steam quality is 0.5 and the initial volume of the vessel is 1 m$^3$. The vessel loses heat at a constant rate $\dot{Q}$ under isobaric conditions so that the quality of steam reduces to 0.1 after 10 hours. The thermodynamic properties of water at 180~$^\circ$C are (subscript $g$: vapor phase; subscript $f$: liquid phase):
			\\
			\begin{tabular}{ll}
				Specific volume: & $v_g = 0.19405$ m$^3$/kg, $v_f = 0.001127$ m$^3$/kg \\
				Specific internal energy: & $u_g = 2583.57$ kJ/kg, $u_f = 762.08$ kJ/kg \\
				Specific enthalpy: & $h_g = 2778.2$ kJ/kg, $h_f = 763.21$ kJ/kg \\
			\end{tabular}
			\\
			The rate of loss $\dot{Q}$ (in kJ/hour) is \rule{3cm}{0.15mm} (rounded off to the nearest integer).

		\item A fractionator recovers 95 mol\% \, \emph{n}-propane as the distillate from an equimolar mixture of \emph{n}-propane and \emph{n}-butane. The condensate is a saturated liquid at 55~$^\circ$C. The Antoine equation is of the form, $\ln(P^{\text{sat}} \text{ [in bar]}) = A - \frac{B}{T + C}$, and the constants are provided below:

			\begin{center}
				\begin{tabular}{|c|c|c|c|}
					\hline
					& A & B & C \\
					\hline
					\emph{n}-propane & 9.1058 & 1872.46 & -25.16 \\
					\emph{n}-butane & 9.0580 & 2154.90 & -34.42 \\
					\hline
				\end{tabular}
			\end{center}

			Assuming Raoult’s law, the condenser pressure (in bar) is \rule{3cm}{0.15mm} (rounded off to one decimal place).

		\item A centrifugal pump is used to pump water (density 1000 kg/m$^3$) from an inlet pressure of $10^5$ Pa to an exit pressure of $2 \times 10^5$ Pa. The exit is at an elevation of 10 m above the pump. The average velocity of the fluid is 10 m/s. The cross-sectional area of the pipe at the pump outlet and inlet is 10$^{-2}$ m$^2$ and acceleration due to gravity is $g = 10$ m/s$^2$. Neglecting losses in the system, the power (in Watts) delivered by the pump is \rule{3cm}{0.15mm} (rounded off to the nearest integer).

		\item A solid sphere of radius 1 cm and initial temperature of 25~$^\circ$C is exposed to a gas stream at 100~$^\circ$C. For the solid sphere, the density is 10$^4$ kg/m$^3$ and the specific heat capacity is 500 J/(kg·K). For the gas, the convective heat transfer coefficient is 1000 W/(m$^2$·K).

			The solid sphere is approximated as a lumped mass system (Biot number \textless{} 0.1). The surface area of the sphere is $4 \pi r^2$. If the thermal conductivity of the gas is 50 W/(m·K), the time (in seconds) needed for the sphere to reach 95~$^\circ$C is \rule{3cm}{0.15mm} (rounded off to the nearest integer).

		\item Stream A with specific heat capacity $C_{pA} = 2000$ J/(kg·K) is cooled from 90~$^\circ$C to 45~$^\circ$C in a concentric double pipe counter current heat exchanger having a heat transfer area of 8 m$^2$. The cold stream B of specific heat capacity $C_{pB} = 1000$ J/(kg·K) enters the exchanger at a flow rate 1 kg/s and 40~$^\circ$C. The overall heat transfer coefficient $U = 250$ W/(m$^2$·K). Assume that the mean driving force is based on the arithmetic mean temperature difference, that is,
			\[ \left[ \Delta T \right]_{LMTD} = \frac{\left( T_{A,in} - T_{B,out} \right)}{2} - \frac{\left( T_{A,out} - T_{B,in} \right)}{2} \]
			where $T_{i,in}$ and $T_{i,out}$ refer to the temperature of the $i$th stream ($i = A, B$) at the inlet and exit, respectively. The mass flow rate of stream A (in kg/s), is \rule{3cm}{0.15mm} (rounded off to two decimal places).

		\item A 20 cm diameter cylindrical solid pellet of a nuclear fuel with density 6000 kg/m$^3$ and conductivity of 300 W/(m·K) generates heat by nuclear fission at a spatially uniform rate of $10^4$ W/kg. The heat from the fuel pellet is transferred to the surrounding coolant by convection such that the pellet wall temperature remains constant at 300~$^\circ$C. Neglecting the axial and azimuthal dependence, the maximum temperature ($T_Q$ in the pellet at steady state is \rule{3cm}{0.15mm} (rounded off to the nearest integer).

		\item The elementary, irreversible, liquid-phase, parallel reactions, $2A \rightarrow D$ and $2A \rightarrow U$, take place in an isothermal non-ideal reactor. The C-curve measured in a tracer experiment is shown in the figure, where $C(t)$ is the concentration of the tracer in g/m$^3$ at the reactor exit at time $t$ (in min).

			\begin{center}{
					\includegraphics[width=0.8\linewidth]{ch_q45.jpg}
			}\end{center}

			The rate constants are $k_1 = 0.2$ Liter/(mol·min) and $k_2 = 0.3$ Liter/(mol·min). Pure A is fed to the reactor at a concentratio

		\item A first-order irreversible liquid phase reaction $A \rightarrow B$ ($k = 0.1$ min$^{-1}$) is carried out under isothermal, steady state conditions in the following reactor arrangement comprising an ideal CSTR (Continuous-Stirred Tank Reactor) and two ideal PFRs (Plug Flow Reactors). From the information in the figure, the volume of the CSTR (in Liters) is \rule{3cm}{0.15mm} (rounded off to the nearest integer).

			\begin{center}{
					\includegraphics[width=0.8\linewidth]{ch_q46.jpg}
			}\end{center}

		\item The elementary liquid-phase irreversible reactions

			\[ A \xrightarrow{k_1 = 0.04\,\text{min}^{-1}} B \xrightarrow{k_2 = 0.1\,\text{min}^{-1}} C \]

			take place in an isothermal ideal CSTR (Continuous-Stirred Tank Reactor). Pure A is fed to the reactor at a concentration of 2 mol/Liter. For the residence time that maximizes the exit concentration of B, the percentage yield of B, defined as \[ \text{\% yield of B} = \left( \frac{\text{net formation rate of B}}{\text{consumption rate of A}} \right) \times 100, \] is \rule{3cm}{0.15mm} (rounded off to the nearest integer).

		\item The elementary irreversible gas-phase reaction $A \rightarrow B + C$ is carried out adiabatically in an ideal CSTR (Continuous-Stirred Tank Reactor) operating at 10 atm. Pure A enters the CSTR at a flow rate of 10 mol/s and a temperature of 450 K. Assume A, B and C to be ideal gases. The specific heat capacity at constant pressure ($C_p$) and heat of formation ($H_f^0$), of component $i$ ( = A, B, C), are:

			\begin{itemize}
				\item $C_{pA} = 30$ J/(mol·K) \quad $H_f^0 = -10$ kJ/mol\\
				\item $C_{pB} = 10$ J/(mol·K) \quad $H_f^0 = 45$ kJ/mol\\
				\item $C_{pC} = 20$ J/(mol·K) \quad $H_f^0 = 5$ kJ/mol\\
			\end{itemize}

			The reaction rate constant $k$ (per second) = $0.1332\exp\left(\frac{-E}{RT}\right)$, where $E = 31.4$ kJ/mol and universal gas constant $R = 0.0821$ atm·L/(mol·K) = 8.314 J/(mol·K). The shaft work and heat losses can be neglected in the analysis, and the gas-phase behavior can be assumed ideal. The reaction temperature is controlled to 728 K. The reactor volume (in Liters) for 75\% conversion is \rule{3cm}{0.15mm} (rounded off to the nearest integer).

		\item For the closed loop system shown in figure, the phase margin (in degrees) is \rule{3cm}{0.15mm} (rounded off to one decimal place).

			\begin{center}{
					\includegraphics[width=0.8\linewidth]{ch_q49.jpg}
			}\end{center}

		\item Two spherical camphor particles of radii 20 cm and 5 cm, far away from each other, are undergoing sublimation in a stream of air. The mass transfer coefficient is proportional to $1/\sqrt{r(t)}$, where $r(t)$ is the radius of the sphere at time $t$. Assume that the partial pressure of camphor far away from the surface of the particle is zero. Also, assume quasi-steady state, identical ambient conditions, and negligible heat effects. If $t_1$ and $t_2$ are the times required for complete sublimation of the 20 cm and 5 cm camphor particles, respectively, the ratio $t_1/t_2$ is \rule{3cm}{0.15mm} (rounded off to one decimal place).

		\item A countercurrent absorption tower is designed to remove 95\% of component A from an incoming binary gas mixture using pure solvent B. The mole ratio of A in the inlet gas is 0.02. The carrier gas flow rate is 50 kmol/h. The equilibrium relation is given by $Y = 2X$, where $Y$ and $X$ are the mole ratios of A in the gas and liquid phases, respectively. If the tower is operated at twice the minimum solvent flow rate, the mole ratio of A in the exit liquid stream is \rule{3cm}{0.15mm} (rounded off to three decimal places).

		\item A binary mixture with components A and B is to be separated in a distillation column to obtain 95 mol\% A as the top product. The binary mixture has a constant relative volatility $\alpha_{AB} = 2$. The column is fed with a saturated liquid containing 50 mol\% A. Under the usual simplifying assumptions such as constant molal overflow, negligible heat loss, ideal trays, the minimum reflux ratio for this separation is \rule{3cm}{0.15mm} (rounded off to one decimal place).

		\item Consider the reactor-separator-recycle process operating under steady state conditions as shown in the figure. The reactor is an ideal Continuous-Stirred Tank Reactor (CSTR), where the reaction $A + B \rightarrow C$ occurs. Assume that there is no impurity in the product and recycle streams. Other relevant information are provided in the figure. The mole fraction of B ($x_B$) in the reactor that minimizes the recycle rate is \rule{3cm}{0.15mm} (rounded off to two decimal places).

			\begin{center}{
					\includegraphics[width=0.8\linewidth]{ch_q53.jpg}
			}\end{center}

		\item Consider two competing equipment A and B. For a compound interest rate of 10\% per annum, in order for equipment B to be the economically cheaper option, its minimum life (in years) is \rule{3cm}{0.15mm} (rounded off to the next higher integer).

			\begin{center}
				\begin{tabular}{|c|c|c|c|}
					Equipment & Capital Cost (Rs) & Yearly Operating Cost (Rs) & Equipment Life (Years) \\
					A & 80,000 & 20,000 & 4 \\
					B & 1,60,000 & 15,000 & ? \\
				\end{tabular}
			\end{center}

		\item A taxi-car is bought for Rs 10 lakhs. Its salvage value is zero. The expected yearly income after paying all expenses and applicable taxes is Rs 3 lakhs. The compound interest rate is 9\% per annum. The discounted payback period (in years), is \rule{3cm}{0.15mm} (rounded off to the next higher integer).

			\begin{center}
				\textbf{END OF QUESTION PAPER }
			\end{center}

	\end{enumerate}

\end{document}




