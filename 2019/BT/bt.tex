\documentclass[12pt]{article}
\usepackage[table]{xcolor}
\usepackage{fancyhdr}
\usepackage{array}
\usepackage{longtable}
\usepackage{booktabs}
\usepackage{amsmath}
\usepackage{multicol}
\usepackage{siunitx}
\usepackage{graphicx}
\usepackage{setspace}
\usepackage{xcolor}
\usepackage{enumitem}
\usepackage{caption}
%\usecaption{subcaption}
\doublespacing
\singlespacing
\usepackage{amssymb}
\usepackage[a4paper, top=1in, bottom=1in, left=1.25in, right=1in]{geometry}
\usepackage{times}
\onehalfspacing  % Or use \singlespacing if that's closer
\pagestyle{fancy}
\fancyhf{}
\renewcommand{\footrulewidth}{0.4pt}
\fancyhead[L]{\textbf{GATE 2019 General Aptitude(GA) Set-8}}
\fancyfoot[L]{GA}
\fancyfoot[R]{\thepage /3}
\setlength{\headheight}{15pt}
\setlength{\headsep}{20pt}
\begin{document}
\noindent\textbf{Q.1 - Q.5 carry one mark each.}\\
\begin{enumerate}[label = Q.\arabic*]
	\item The fisherman, \underline{\hspace{2cm}} the flood victims owed their lives, were rewarded by the government.
		\begin{multicols}{4}
			\begin{enumerate}[label=(\Alph*)]
				\item whom
				\item to which
				\item to whom
				\item that
			\end{enumerate}
		\end{multicols}
	\item Some students were not involved in the strike \\[1em]
		If the above statement is true, which of the following conclusions is/are logically necessary?
		\begin{enumerate}[label=\arabic*.]
			\item Some who were involved in the strike were students 
			\item No student was involved in the strike.
			\item At least one student was involved in the strike.
			\item Some who were not involved in the strike were students.
		\end{enumerate}
		\begin{multicols}{4}
			\begin{enumerate}[label=(\Alph*)]
				\item 1 and 2 \item 3 \item 4 \item 2 and 3
			\end{enumerate}
		\end{multicols}
	\item The radius as well as the height of a circular cone increases by 10\% The percentage increase in its volume is \underline{\hspace{2cm}}.
		\begin{multicols}{4}
			\begin{enumerate}[label=(\Alph*)]
				\item 17.1
				\item 21.0
				\item 33.1
				\item 72.8
			\end{enumerate}
		\end{multicols}
	\item Five numbers 10,7,5,4 and 2 are to be arranged in a sequence from left to right following the directions given below:
		\begin{enumerate}[label=\arabic*.]
			\item No two odd or even numbers are next to each other.
			\item The second number from the left is exactly half of the left most-number.
			\item The middle number is exactly twice the right-most number.
		\end{enumerate}
		Which is the second number from the right?
		\begin{multicols}{4}
			\begin{enumerate}[label=(\Alph*)]
				\item 2 \item 4 \item 7 \item 10
			\end{enumerate}
		\end{multicols}
	\item Until Iran came along , India had never been \underline{\hspace{2cm}} in kabaddi.
		\begin{multicols}{4}
			\begin{enumerate}[label=(\Alph*)]
				\item defeated \item defeating \item defeat \item defeatist
			\end{enumerate}
		\end{multicols}
		\newpage
\end{enumerate}
\noindent\textbf{Q.6 - Q.10 carry two marks each.}
\begin{enumerate}[start=6,label=Q.\arabic*]
	\item Since the last one year, after a 125 basis point reduction in rpo rate by the Reserve Bank of India, banking institutes have been making a demand to reduce intrest rates on small saving schemes. Finally, the government announced yesterday a reduction in interest rates on small saving schemes to bring them on par with fixed deposit interest rates.\\
		Which one of the following statements can be inferred from the given passage?
		\begin{enumerate}[label=(\Alph*)]
			\item Whenever the Reserve Bank of India reduces the repo rate, the interest rates on small saving schemes are also reduced
			\item Interest rates on small saving schemes are always maintained on par with fixed  deposit interest rates 
			\item The government sometimes takes into consideration the demands of  banking institutions before reducing interest rates on small saving schemes
			\item A reduction in interest rates on small saving schemes follow only after a reduction in repo rate by the Reserve Bank of India
		\end{enumerate}
	\item In a country of 1400 million population, 70\% own mobiles. Among the mobile phone owners, only 294million access the Internet.Among the Internet users, only half buy goods from e-commerce portals.What is the percentage of these buyers in the country?
		\begin{multicols}{4}
			\begin{enumerate}[label=(\Alph*)]
				\item 10.50 \item 14.70 \item 15.00 \item 50.00
			\end{enumerate}
		\end{multicols}
	\item The nomenclature of the hindustani music has changed over the centuries.Since the medieval period \textit{dhrupad} styles were identified as \textit{baanis}. Terms like \textit{gayaki} and \textit{baaj} were used to refer to vocal and instrumental styles, respectively.With the instrumentalization of music education the terms \textit{gharana} became acceptable.\textit{Gharana} originally referred to hereditary musicians from a particular lineage,including disciples and grand disciples.\\
		Which of the following pairings is NOT correct?
		\begin{enumerate}[label=(\Alph*)]
			\item \textit{dhrupad,baani}
			\item \textit{gayaki},vocal
			\item \textit{baaj},institution
			\item \textit{gharana},lineage
		\end{enumerate}
		\vfill
		\newpage
	\item Two trains started at 7AM from same point. The first train travelled north at a speed of 80km/h and the second train travelled at speed of 100km/h. The time at which they were 540 km apart is \underline{\hspace{2cm}}AM.
		\begin{multicols}{4}
			\begin{enumerate}[label=(\Alph*)]
				\item 9 \item 10 \item 11 \item 11.30
			\end{enumerate}
		\end{multicols}
	\item "I read somewhere that in ancient times the prestige of a kingdom depended upon the number of taxes that it was able to levy on its people. It was very much like the prestige of a head-hunter in his own community."\\
		Based on the paragraph above, the prestige of a head-hunter depended upon \underline{\hspace{2cm}}
		\begin{enumerate}[label=(\Alph*)]
			\item the prestige of a kingdom
			\item the prestige of the heads
			\item the number of taxes he could levy
			\item the number of heads he could gather
		\end{enumerate}
		\vspace{\baselineskip}
		\begin{center}
			{\textbf{END OF THE QUESTION PAPER}}
		\end{center}
		\newpage
\end{enumerate}
\fancyhead[L]{GATE 2019}
\fancyhead[R]{Biotechnology}
\fancyfoot[L]{BT}
\setcounter{page}{1}
\fancyfoot[R]{\thepage/11}
\noindent\textbf{Q.1 - Q.25 carry one mark each.}
\begin{enumerate}[label=Q.\arabic*]
	\item The Bt toxin gene from \textit{Bacillus thuringiensis} used to generate genetically modified crops is
		\begin{multicols}{4}
			\begin{enumerate}[label=(\Alph*)]
				\item \textit{cry}
				\item \textit{cro}
				\item \textit{cdc}
				\item \textit{cre}
			\end{enumerate}
		\end{multicols}

	\item Which one of the following is used as a pH indicator in animal cell culture medium?
		\begin{multicols}{2}
			\begin{enumerate}[label=(\Alph*)]
				\item Acridine orange
				\item Phenol red
				\item Bromophenol blue
				\item Coomassie blue
			\end{enumerate}
		\end{multicols}

	\item Tetracycline inhibits the
		\begin{enumerate}[label=(\Alph*)]
			\item interaction between tRNA and mRNA 
			\item translocation of mRNA through ribosome 
			\item peptidyl transferase activity 
			\item binding of amino-acyl tRNA to ribosome
		\end{enumerate}

	\item Which one of the following is a database of protein sequence motifs?
		\begin{multicols}{4}
			\begin{enumerate}[label=(\Alph*)]
				\item PROSITE
				\item TrEMBL
				\item SWISSPROT
				\item PDB
			\end{enumerate}
		\end{multicols}
	\item Which one of the following enzymes is encoded by human immunodeficiency virus (HIV) genome?
		\begin{multicols}{2}
			\begin{enumerate}[label=(\Alph*)]
				\item Reverse transcriptase
				\item Phospholipase
				\item Phosphatase
				\item ATP synthase
			\end{enumerate}
		\end{multicols}

	\item DNA synthesis in eukaryotes occurs during which phase of the mitotic cell cycle?
		\begin{multicols}{4}
			\begin{enumerate}[label=(\Alph*)]
				\item M
				\item G$_1$
				\item S
				\item G$_0$
			\end{enumerate}
		\end{multicols}

	\item Match the human diseases in Group I with the causative agents in Group II.

		\begin{center}
			\begin{tabular}{ll}
				\textbf{Group I} & \textbf{Group II} \\
				P. Amoebiasis & 1. \textit{Leishmania donovani} \\
				Q. African sleeping sickness & 2. \textit{Trypanosoma cruzi} \\
				R. Kala azar & 3. \textit{Entamoeba histolytica} \\
				S. Chagas’ disease & 4. \textit{Trypanosoma gambiense} \\
			\end{tabular}
		\end{center}

		\begin{multicols}{2}
			\begin{enumerate}[label=(\Alph*)]
				\item P-3, Q-4, R-2, S-1 
				\item P-3, Q-2, R-1, S-4 
				\item P-3, Q-4, R-1, S-2 
				\item P-4, Q-3, R-1, S-2
			\end{enumerate}
		\end{multicols}

	\item Which one of the following techniques can be used to compare the expression of a large number of genes in two biological samples in a single experiment?
		\begin{multicols}{2}
			\begin{enumerate}[label=(\Alph*)]
				\item Polymerase chain reaction
				\item DNA microarray
				\item Northern hybridization
				\item Southern hybridization
			\end{enumerate}
		\end{multicols}
	\item Which of the following processes can increase genetic diversity of bacteria in nature?

		P. Conjugation \\
		Q. Transformation \\
		R. Transduction \\
		S. Transfection

		\begin{multicols}{2}
			\begin{enumerate}[label=(\Alph*)]
				\item P only 
				\item P and Q only 
				\item P, Q and R only 
				\item P, Q, R and S
			\end{enumerate}
		\end{multicols}

	\item Which one of the following is NOT a part of the human nonspecific defense system?
		\begin{multicols}{4}
			\begin{enumerate}[label=(\Alph*)]
				\item Interferon
				\item Mucous
				\item Saliva
				\item Antibody
			\end{enumerate}
		\end{multicols}

	\item A mutation in a gene that codes for a polypeptide results in a variant polypeptide that lacks the last three amino acids. What type of mutation is this?
		\begin{multicols}{2}
			\begin{enumerate}[label=(\Alph*)]
				\item Synonymous mutation 
				\item Nonsense mutation 
				\item Missense mutation 
				\item Silent mutation
			\end{enumerate}
		\end{multicols}

	\item Which one of the following equations represents a one-dimensional wave equation?

		\begin{multicols}{4}
			\begin{enumerate}[label=(\Alph*)]
				\item $\frac{\partial u}{\partial t} = c^2 \frac{\partial^2 u}{\partial x^2}$ \\
				\item $\frac{\partial^2 u}{\partial t^2} = c^2 \frac{\partial u}{\partial x}$ \\
				\item $\frac{\partial^2 u}{\partial t^2} = c^2 \frac{\partial^2 u}{\partial x^2}$ \\
				\item $\frac{\partial^2 u}{\partial t^2} + \frac{\partial^2 u}{\partial x^2} = 0$
			\end{enumerate}
		\end{multicols}
	\item Which of the following are geometric series?

		P. \( 1, 6, 11, 16, 21, 26, \ldots \) \\
		Q. \( 9, 6, 3, 0, -3, -6, \ldots \) \\
		R. \( 1, 3, 9, 27, 81, \ldots \) \\
		S. \( 4, -8, 16, -32, 64, \ldots \)

		\begin{multicols}{2}
			\begin{enumerate}[label=(\Alph*)]
				\item P and Q only 
				\item R and S only 
				\item P, Q and S only 
				\item P, Q and R only
			\end{enumerate}
		\end{multicols}

	\item Which one of the following statements is CORRECT for enzyme catalyzed reactions? (\( \Delta G \) is Gibbs free energy change, \( K_{eq} \) is equilibrium constant)

		\begin{multicols}{2}
			\begin{enumerate}[label=(\Alph*)]
				\item Enzymes affect \( \Delta G \), but not \( K_{eq} \) 
				\item Enzymes affect \( K_{eq} \), but not \( \Delta G \) 
				\item Enzymes affect both \( \Delta G \) and \( K_{eq} \) 
				\item Enzymes do not affect \( \Delta G \) or \( K_{eq} \)
			\end{enumerate}
		\end{multicols}

	\item Which one of the following can NOT be a limiting substrate if Monod’s growth kinetics is applicable?

		\begin{multicols}{2}
			\begin{enumerate}[label=(\Alph*)]
				\item Extracellular carbon source 
				\item Extracellular nitrogen source 
				\item Dissolved oxygen 
				\item Intracellular carbon source
			\end{enumerate}
		\end{multicols}

	\item Which one of the following is the unit of heat transfer coefficient?

		\begin{multicols}{2}
			\begin{enumerate}[label=(\Alph*)]
				\item W m\textsuperscript{-2} K\textsuperscript{-1} 
				\item W m\textsuperscript{2} K 
				\item W m\textsuperscript{-2} K\textsuperscript{1} 
				\item W m\textsuperscript{2} K\textsuperscript{-1}
			\end{enumerate}
		\end{multicols}
	\item Which one of the following is catabolized during endogenous metabolism in a batch bacterial cultivation?

		\begin{multicols}{2}
			\begin{enumerate}[label=(\Alph*)]
				\item internal reserves 
				\item extracellular substrates 
				\item extracellular products 
				\item toxic substrates
			\end{enumerate}
		\end{multicols}

	\item Which one of the following need NOT be conserved in a biochemical reaction?

		\begin{multicols}{2}
			\begin{enumerate}[label=(\Alph*)]
				\item Total mass 
				\item Total moles 
				\item Number of atoms of each element 
				\item Total energy
			\end{enumerate}
		\end{multicols}

	\item The number of possible rooted trees in a phylogeny of three species is \_\_\_\_\_\_\_\_.

	\item Matrix \( A = \begin{bmatrix} 0 & 6 \\ p & 0 \end{bmatrix} \) will be skew-symmetric when \( p = \_\_\_\_\_\_\_\_ \).
		\item The solution of \( \lim_{x \to 8} \left( \frac{x^2 - 64}{x - 8} \right) \) is \_\_\_\_\_\_\_\_.

		\item The median value for the dataset \( (12, 10, 16, 8, 90, 50, 30, 24) \) is \_\_\_\_\_\_\_\_.

		\item The degree of reduction for acetic acid (\(\mathrm{C_2H_4O_2}\)) is \_\_\_\_\_\_\_\_.

		\item The mass of 1 kmol of oxygen molecules is \_\_\_\_\_\_ g (rounded off to the nearest integer).
		\item Protein concentration of a crude enzyme preparation was \(10\, \mathrm{mg\, mL^{-1}}\). \(10\, \mu\mathrm{L}\) of this sample gave an activity of \(5\, \mu\mathrm{mol\, min^{-1}}\) under standard assay conditions. The specific activity of this crude enzyme preparation is \_\_\_\_\_\_\_\_ units \( \mathrm{mg^{-1}} \).
\end{enumerate}
\textbf{Q.26 - Q.55 carry two marks each}
\begin{enumerate}[label=Q.\arabic*,start=26]
	\item In general, which one of the following statements is NOT CORRECT?
		\begin{enumerate}[label=(\Alph*)]
			\item Hydrogen bonds result from electrostatic interactions
			\item Hydrogen bonds contribute to the folding energy of proteins
			\item Hydrogen bonds are weaker than van der Waals interactions
			\item Hydrogen bonds are directional
		\end{enumerate}

	\item For site-directed mutagenesis, which one of the following restriction enzymes is used to digest methylated DNA?
		\begin{enumerate}[label=(\Alph*)]
			\item KpnI
			\item DpnI
			\item XhoI
			\item MluI
		\end{enumerate}

	\item Match the organelles in Group I with their functions in Group II.

		\begin{center}
			\begin{tabular}{ll}
				\textbf{Group I} & \textbf{Group II} \\
				P. Lysosome & 1. Digestion of foreign substances \\
				Q. Smooth ER & 2. Protein targeting \\
				R. Golgi apparatus & 3. Lipid synthesis \\
				S. Nucleolus & 4. Protein synthesis \\
				& 5. rRNA synthesis \\
			\end{tabular}
		\end{center}

		\begin{enumerate}[label=(\Alph*)]
			\item P-1, Q-3, R-2, S-5
			\item P-1, Q-4, R-5, S-3
			\item P-2, Q-5, R-3, S-4
			\item P-1, Q-3, R-4, S-5
		\end{enumerate}

	\item Which of the following statements are \textbf{CORRECT} when a protein sequence database is searched using the BLAST algorithm?

		\begin{itemize}
			\item[P.] A larger E-value indicates higher sequence similarity
			\item[Q.] E-value $< 10^{-10}$ indicates sequence homology
			\item[R.] A higher BLAST score indicates higher sequence similarity
			\item[S.] E-value $> 10^{10}$ indicates sequence homology
		\end{itemize}

		\begin{enumerate}[label=(\Alph*)]
			\item P, Q and R only
			\item Q and R only
			\item P, R and S only
			\item P and S only
		\end{enumerate}
	\item Which one of the following is coded by the ABO blood group locus in the human genome?
		\begin{enumerate}[label=(\Alph*)]
			\item Acyl transferase
			\item Galactosyltransferase
			\item Transposase
			\item $\beta$-Galactosidase
		\end{enumerate}

	\item Which of the following factors affect the fidelity of DNA polymerase in polymerase chain reaction?
		\begin{itemize}
			\item[P.] Mg$^{2+}$ concentration
			\item[Q.] pH
			\item[R.] Annealing temperature
		\end{itemize}

		\begin{enumerate}[label=(\Alph*)]
			\item P and Q only
			\item P and R only
			\item Q and R only
			\item P, Q and R
		\end{enumerate}

	\item Group I lists spectroscopic methods and Group II lists biomolecular applications of these methods. Match the methods in Group I with the applications in Group II.

		\begin{center}
			\begin{tabular}{ll}
				\textbf{Group I} & \textbf{Group II} \\
				P. Infrared & 1. Identification of functional groups \\
				Q. Circular Dichroism & 2. Determination of secondary structure \\
				R. Nuclear Magnetic Resonance & 3. Estimation of molecular weight \\
				& 4. Determination of 3-D structure \\
			\end{tabular}
		\end{center}
	\item The hexapeptide P has an isoelectric point (pI) of 6.9. Hexapeptide Q is a variant of P that contains valine instead of glutamate at position 3. The two peptides are analyzed by polyacrylamide gel electrophoresis at pH 8.0. Which one of the following statements is CORRECT?
		\begin{enumerate}[label=(\Alph*)]
			\item P will migrate faster than Q towards the anode
			\item P will migrate faster than Q towards the cathode
			\item Both P and Q will migrate together
			\item Q will migrate faster than P towards the anode
		\end{enumerate}

	\item Antibody-producing hybridoma cells are generated by the fusion of a
		\begin{enumerate}[label=(\Alph*)]
			\item T cell with a myeloma cell
			\item B cell with a myeloma cell
			\item Macrophage with a myeloma cell
			\item T cell and a B cell
		\end{enumerate}

	\item Which of the following statements are CORRECT about the function of fetal bovine serum in animal cell culture?

		\begin{itemize}
			\item[P.] It stimulates cell growth
			\item[Q.] It enhances cell attachment
			\item[R.] It provides hormones and minerals
			\item[S.] It maintains pH at 7.4
		\end{itemize}

		\begin{enumerate}[label=(\Alph*)]
			\item P and Q only
			\item P and S only
			\item P, Q and R only
			\item P, Q, R and S
		\end{enumerate}

	\item Which one of the following covalent linkages exists between 7-Methyl guanosine (m\textsuperscript{7}G) and mRNAs?
		\begin{enumerate}[label=(\Alph*)]
			\item 2$'$-3$'$ triphosphate
			\item 2$'$-5$'$ triphosphate
			\item 5$'$-5$'$ triphosphate
			\item 5$'$-2$'$ triphosphate
		\end{enumerate}
	\item Which one of the following amino acid residues will destabilize an $\alpha$-helix when inserted in the middle of the helix?
		\begin{enumerate}[label=(\Alph*)]
			\item Pro
			\item Val
			\item Ile
			\item Leu
		\end{enumerate}

	\item What is the solution of the differential equation $\frac{dy}{dx} = \frac{x}{y}$, with the initial condition, at $x = 0$, $y = 1$?
		\begin{enumerate}[label=(\Alph*)]
			\item $x^2 = y^2 + 1$
			\item $y^2 = x^2 + 1$
			\item $y^2 = 2x^2 + 1$
			\item $x^2 - y^2 = 0$
		\end{enumerate}

	\item The Laplace transform of the function $f(t) = t^2 + 2t + 1$ is
		\begin{enumerate}[label=(\Alph*)]
			\item $\frac{1}{s^3} + \frac{3}{s^2} + \frac{1}{s}$
			\item $\frac{4}{s^3} + \frac{4}{s^2} + \frac{1}{s}$
			\item $\frac{1}{s^3} + \frac{2}{s^2} + \frac{1}{s}$
			\item $\frac{2}{s^3} + \frac{2}{s^2} + \frac{1}{s}$
		\end{enumerate}

	\item Which of the following factors can influence the lag phase of a microbial culture in a batch fermentor?

		\begin{itemize}
			\item[P.] Inoculum size
			\item[Q.] Inoculum age
			\item[R.] Medium composition
		\end{itemize}

		\begin{enumerate}[label=(\Alph*)]
			\item P and Q only
			\item Q and R only
			\item P and R only
			\item P, Q and R
		\end{enumerate}
	\item Which one of the following statements is CORRECT about proportional controllers?
		\begin{enumerate}[label=(\Alph*)]
			\item The initial change in control output signal is relatively slow
			\item The initial corrective action is greater for larger error
			\item They have no offset
			\item There is no corrective action if the error is a constant
		\end{enumerate}

	\item Match the instruments in Group I with their corresponding measurements in Group II.

		\begin{center}
			\begin{tabular}{ll}
				\textbf{Group I} & \textbf{Group II} \\
				P. Manometer & 1. Agitator speed \\
				Q. Rotameter & 2. Pressure difference \\
				R. Tachometer & 3. Cell number \\
				S. Haemocytometer & 4. Air flow rate \\
			\end{tabular}
		\end{center}

		\begin{enumerate}[label=(\Alph*)]
			\item P-4, Q-1, R-2, S-3
			\item P-3, Q-4, R-1, S-2
			\item P-2, Q-4, R-1, S-3
			\item P-2, Q-1, R-4, S-3
		\end{enumerate}

	\item Which of the following statements is ALWAYS CORRECT about an ideal chemostat?

		\begin{itemize}
			\item[P.] Substrate concentration inside the chemostat is equal to that in the exit stream
			\item[Q.] Optimal dilution rate is lower than critical dilution rate
			\item[R.] Biomass concentration increases with increase in dilution rate
			\item[S.] Cell recirculation facilitates operation beyond critical dilution rate
		\end{itemize}

		\begin{enumerate}[label=(\Alph*)]
			\item P and Q only
			\item P, R and S only
			\item P and S only
			\item P, Q and S only
		\end{enumerate}
		\newpage
	\item Determine the correctness or otherwise of the following Assertion [a] and the Reason [r]:

		\textbf{Assertion [a]}: It is possible to regenerate a whole plant from a single plant cell. \\
		\textbf{Reason [r]}: It is easier to introduce transgenes in to plants than animals.
		\begin{enumerate}[label=(\Alph*)]
			\item Both [a] and [r] are true and [r] is the correct reason for [a]
			\item Both [a] and [r] are true but [r] is not the correct reason for [a]
			\item Both [a] and [r] are false
			\item Only [a] is true but [r] is false
		\end{enumerate}

	\item A UV-visible spectrophotometer has a minimum detectable absorbance of 0.02. The minimum concentration of a protein sample that can be measured reliably in this instrument with a cuvette of 1 cm path length is \underline{\hspace{2cm}} µM. The molar extinction coefficient of the protein is 10,000 L mol$^{-1}$ cm$^{-1}$.

	\item The difference in concentrations of an uncharged solute between two compartments is 1.6-fold. The energy required for active transport of the solute across the membrane separating the two compartments is \underline{\hspace{2cm}} cal mol$^{-1}$ (rounded off to the nearest integer). (R = 1.987 cal mol$^{-1}$ K$^{-1}$, T = 37 $^\circ$C)

	\item In pea plants, purple color of flowers is determined by the dominant allele while white color is determined by the recessive allele. A genetic cross between two purple flower-bearing plants results in an offspring with white flowers. The probability that the third offspring from these parents will have purple flowers is \underline{\hspace{2cm}} (rounded off to 2 decimal places).

	\item The molecular mass of a protein is 22 kDa. The size of the cDNA (excluding the untranslated regions) that codes for this protein is \underline{\hspace{2cm}} kb (rounded off to 1 decimal place).
	\item A new game is being introduced in a casino. A player can lose Rs. 100, break even, win Rs. 100, or win Rs. 500. The probabilities $P(X)$ of each of these outcomes $X$ are given in the following table:

		\begin{center}
			\begin{tabular}{|c|c|c|c|c|}
				\hline
				$X$ (in Rs.) & -100 & 0 & 100 & 500 \\
				\hline
				$P(X)$ & 0.25 & 0.5 & 0.2 & 0.05 \\
				\hline
			\end{tabular}
		\end{center}

		The standard deviation $\sigma$ for the casino payout is Rs. \underline{\hspace{2cm}} (rounded off to the nearest integer).

	\item $\int_{-1}^{1} f(x) dx$ calculated using trapezoidal rule for the values given in the table is \underline{\hspace{2cm}} (rounded off to 2 decimal places).

		\begin{center}
			\begin{tabular}{|c|c|c|c|c|c|c|c|}
				\hline
				$x$ & $-1$ & $-2/3$ & $-1/3$ & $0$ & $1/3$ & $2/3$ & $1$ \\
				\hline
				$f(x)$ & 0.37 & 0.51 & 0.71 & 1.10 & 1.40 & 1.95 & 2.71 \\
				\hline
			\end{tabular}
		\end{center}

	\item Yeast biomass (C$_6$H$_{10}$O$_3$N) grown on glucose is described by the stoichiometric equation given below: \\
		C$_6$H$_{12}$O$_6$ + 0.48 NH$_3$ + 3 O$_2$ $\rightarrow$ 0.48 C$_6$H$_{10}$O$_3$N + 3.12 CO$_2$ + 4.32 H$_2$O

		The amount of glucose needed for the production of 50 g L$^{-1}$ of yeast biomass in a batch reactor with a working volume of 1,00,000 L is \underline{\hspace{2cm}} kg (rounded off to the nearest integer).

	\item Phenolic wastewater discharged from an industry was treated with \textit{Pseudomonas} sp. in an aerobic bioreactor. The influent and effluent concentrations of phenol were 10,000 and 10 ppm, respectively. The inlet feed rate of wastewater was 80 L h$^{-1}$. The kinetic properties of the organism are as follows:

		\begin{itemize}
			\item Maximum specific growth rate ($\mu_m$) = 1 h$^{-1}$
			\item Saturation constant ($K_S$) = 100 mg L$^{-1}$
			\item Cell death rate ($k_d$) = 0.01 h$^{-1}$
		\end{itemize}

		Assuming that the bioreactor operates under ‘chemostat’ mode, the working volume required for this process is \underline{\hspace{2cm}} L (rounded off to the nearest integer).
	\item In a cross-flow filtration process, the pressure drop ($\Delta P$) driving the fluid flow is 2 atm, inlet feed pressure ($P_i$) is 3 atm and filtrate pressure ($P_f$) is equal to atmospheric pressure. The average transmembrane pressure drop ($\Delta P_m$) is \underline{\hspace{2cm}} atm.

	\item An industrial fermentor containing 10,000 L of medium needs to be sterilized. The initial spore concentration in the medium is $10^6$ spores mL$^{-1}$. The desired probability of contamination after sterilization is $10^{-3}$. The death rate of spores at 121 °C is 4 min$^{-1}$. Assume that there is no cell death during heating and cooling phases. The holding time of the sterilization process is \underline{\hspace{2cm}} min (rounded off to the nearest integer).

	\item The dimensions and operating condition of a lab-scale fermentor are as follows:
		\begin{itemize}
			\item Volume = 1 L
			\item Diameter = 20 cm
			\item Agitator speed = 600 rpm
			\item Ratio of impeller diameter to fermentor diameter = 0.3
		\end{itemize}

		This fermentor needs to be scaled up to 8,000 L for a large-scale industrial application. If the scale-up is based on constant impeller tip speed, the speed of the agitator in the larger reactor is \underline{\hspace{2cm}} rpm. Assume that the scale-up factor is the cube root of the ratio of fermentor volumes.
		\begin{center}{
				\textbf{END OF THE QUESTION PAPER
				}
		}\end{center}
\end{enumerate}
\end{document}
