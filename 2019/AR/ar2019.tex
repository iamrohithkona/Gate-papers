\documentclass[12pt]{article}
\usepackage{longtable}
\usepackage{graphicx}
\usepackage{fancyhdr}
\usepackage[table]{xcolor}
\usepackage{amsmath}
\usepackage{multicol}
\usepackage{siunitx}
\usepackage{setspace}
\usepackage{xcolor}
\usepackage{enumitem}
\usepackage{caption}
%\usecaption{subcaption}
\doublespacing
\usepackage{array}
\setlength\emergencystretch{1em}
\singlespacing
\usepackage{amssymb}
\usepackage[a4paper, top=1in, bottom=1in, left=1.25in, right=1in]{geometry}
\usepackage{times}
\onehalfspacing  % Or use \singlespacing if that's closer
\pagestyle{fancy}
\fancyhf{}
\renewcommand{\footrulewidth}{0.4pt}
\fancyhead[L]{\textbf{GATE 2019 General Aptitude(GA) Set-8}}
\fancyfoot[L]{GA}
\fancyfoot[R]{\thepage /3}
\setlength{\headheight}{15pt}
\setlength{\headsep}{20pt}
\begin{document}
\noindent\textbf{Q.1 - Q.5 carry one mark each.}\hfill
\begin{enumerate}[label = Q.\arabic*]
	\item The fisherman, \underline{\hspace{2cm}} the flood victims owed their lives, were rewarded by the government.
		\begin{multicols}{4}
			\begin{enumerate}[label=(\Alph*)]
				\item whom
				\item to which
				\item to whom
				\item that
			\end{enumerate}
		\end{multicols}
	\item Some students were not involved in the strike \\[1em]
		If the above statement is true, which of the following conclusions is/are logically necessary?
		\begin{enumerate}[label=\arabic*.]
			\item Some who were involved in the strike were students 
			\item No student was involved in the strike.
			\item At least one student was involved in the strike.
			\item Some who were not involved in the strike were students.
		\end{enumerate}
		\begin{multicols}{4}
			\begin{enumerate}[label=(\Alph*)]
				\item 1 and 2 \item 3 \item 4 \item 2 and 3
			\end{enumerate}
		\end{multicols}
	\item The radius as well as the height of a circular cone increases by 10\% The percentage increase in its volume is \underline{\hspace{2cm}}.
		\begin{multicols}{4}
			\begin{enumerate}[label=(\Alph*)]
				\item 17.1
				\item 21.0
				\item 33.1
				\item 72.8
			\end{enumerate}
		\end{multicols}
	\item Five numbers 10,7,5,4 and 2 are to be arranged in a sequence from left to right following the directions given below:
		\begin{enumerate}[label=\arabic*.]
			\item No two odd or even numbers are next to each other.
			\item The second number from the left is exactly half of the left most-number.
			\item The middle number is exactly twice the right-most number.
		\end{enumerate}
		Which is the second number from the right?
		\begin{multicols}{4}
			\begin{enumerate}[label=(\Alph*)]
				\item 2 \item 4 \item 7 \item 10
			\end{enumerate}
		\end{multicols}
	\item Until Iran came along , India had never been \underline{\hspace{2cm}} in kabaddi.
		\begin{multicols}{4}
			\begin{enumerate}[label=(\Alph*)]
				\item defeated \item defeating \item defeat \item defeatist
			\end{enumerate}
		\end{multicols}
		\newpage
\end{enumerate}
\noindent\textbf{Q.6 - Q.10 carry two marks each.}
\begin{enumerate}[start=6,label=Q.\arabic*]
	\item Since the last one year, after a 125 basis point reduction in rpo rate by the Reserve Bank of India, banking institutes have been making a demand to reduce intrest rates on small saving schemes. Finally, the government announced yesterday a reduction in interest rates on small saving schemes to bring them on par with fixed deposit interest rates.\\
		Which one of the following statements can be inferred from the given passage?
		\begin{enumerate}[label=(\Alph*)]
			\item Whenever the Reserve Bank of India reduces the repo rate, the interest rates on small saving schemes are also reduced
			\item Interest rates on small saving schemes are always maintained on par with fixed  deposit interest rates 
			\item The government sometimes takes into consideration the demands of  banking institutions before reducing interest rates on small saving schemes
			\item A reduction in interest rates on small saving schemes follow only after a reduction in repo rate by the Reserve Bank of India
		\end{enumerate}
	\item In a country of 1400 million population, 70\% own mobiles. Among the mobile phone owners, only 294million access the Internet.Among the Internet users, only half buy goods from e-commerce portals.What is the percentage of these buyers in the country?
		\begin{multicols}{4}
			\begin{enumerate}[label=(\Alph*)]
				\item 10.50 \item 14.70 \item 15.00 \item 50.00
			\end{enumerate}
		\end{multicols}
	\item The nomenclature of the hindustani music has changed over the centuries.Since the medieval period \textit{dhrupad} styles were identified as \textit{baanis}. Terms like \textit{gayaki} and \textit{baaj} were used to refer to vocal and instrumental styles, respectively.With the instrumentalization of music education the terms \textit{gharana} became acceptable.\textit{Gharana} originally referred to hereditary musicians from a particular lineage,including disciples and grand disciples.\\
		Which of the following pairings is NOT correct?
		\begin{enumerate}[label=(\Alph*)]
			\item \textit{dhrupad,baani}
			\item \textit{gayaki},vocal
			\item \textit{baaj},institution
			\item \textit{gharana},lineage
		\end{enumerate}
		\vfill
		\newpage
	\item Two trains started at 7AM from same point. The first train travelled north at a speed of 80km/h and the second train travelled at speed of 100km/h. The time at which they were 540 km apart is \underline{\hspace{2cm}}AM.
		\begin{multicols}{4}
			\begin{enumerate}[label=(\Alph*)]
				\item 9 \item 10 \item 11 \item 11.30
			\end{enumerate}
		\end{multicols}
	\item "I read somewhere that in ancient times the prestige of a kingdom depended upon the number of taxes that it was able to levy on its people. It was very much like the prestige of a head-hunter in his own community."\hfill
		Based on the paragraph above, the prestige of a head-hunter depended upon \underline{\hspace{2cm}}
		\begin{enumerate}[label=(\Alph*)]
			\item the prestige of a kingdom
			\item the prestige of the heads
			\item the number of taxes he could levy
			\item the number of heads he could gather
		\end{enumerate}
		\vspace{\baselineskip}
		\begin{center}
			{\textbf{END OF THE QUESTION PAPER}}
		\end{center}
		\newpage
\end{enumerate}
\fancyhead[L]{GATE 2019}
\fancyhead[R]{Architecture and Planning}
\fancyfoot[L]{AR}
\setcounter{page}{1}
\fancyfoot[R]{\thepage/13}
\noindent\textbf{Q.1 - Q.25 carry one mark each.}
\begin{enumerate}[label=Q.\arabic*]
	\item Which of the following commands in AUTOCAD is used to create 3D solid between various cross sections?
		\begin{multicols}{4}
			\begin{enumerate}
				\item LOFT
				\item MESH
				\item XEDGES
				\item PFACE
			\end{enumerate}
		\end{multicols}

	\item Name the architect who criticized ornament in useful objects in his essay \textit{‘Ornament and Crime’}.
		\begin{multicols}{2}
			\begin{enumerate}
				\item John Ruskin
				\item H P Berlage
				\item Adolf Loos
				\item Walter Gropius
			\end{enumerate}
		\end{multicols}

	\item A sanitary landfill is provided with High Density Poly Ethylene (HDPE) lining along the ground surface. This is provided primarily to prevent
		\begin{multicols}{2}
			\begin{enumerate}
				\item Bleaching
				\item Leaching
				\item Rodents
				\item Plant growth
			\end{enumerate}
		\end{multicols}

	\item Super-elevation of a road with pre-determined radius of curvature is primarily dependent on
		\begin{enumerate}
			\item Altitude
			\item Soil bearing capacity
			\item Traffic volume
			\item Design traffic speed
		\end{enumerate}

	\item In a mono-centric urban model, land rent is expected to
		\begin{enumerate}
			\item diminish as one moves towards the center
			\item diminish as one moves away from the center
			\item remain constant across the whole urban area
			\item be unrelated with distance from center
		\end{enumerate}

	\item Fineness modulus of sand measures its
		\begin{enumerate}
			\item Compressive strength
			\item Grading according to particle size
			\item Bulking of sand
			\item Ratio of coarse and fine sand
		\end{enumerate}

	\item The spherical surface of the geodesic dome comprises of
		\begin{enumerate}
			\item Equilateral triangles of various sizes
			\item Isosceles triangles of various sizes
			\item Equilateral triangles of uniform size
			\item Isosceles triangles of uniform size
		\end{enumerate}

	\item The abrupt change or junction between two ecological zones is termed as
		\begin{multicols}{2}
			\begin{enumerate}
				\item Ecological niche
				\item Ecosystem
				\item Ecotype
				\item Ecotone
			\end{enumerate}
		\end{multicols}

	\item Complementary colours in a Munsell pigment colour wheel refers to
		\begin{multicols}{2}
			\begin{enumerate}
				\item Colours in alternate positions
				\item Colours opposite to one another
				\item Colours adjacent to each other
				\item A pair of secondary colours
			\end{enumerate}
		\end{multicols}

	\item The closing syntax, for an executable command line in C or C++ program, is
		\begin{multicols}{4}
			\begin{enumerate}
				\item :
				\item ,
				\item ;
				\item .
			\end{enumerate}
		\end{multicols}

	\item The term ‘Necropolis’ refers to
		\begin{multicols}{2}
			\begin{enumerate}
				\item Organically growing settlement
				\item Origin of a settlement
				\item A dead settlement
				\item Merging of two settlements
			\end{enumerate}
		\end{multicols}

	\item Which of the following projection types is adopted in the Universal Transverse Mercator (UTM)?
		\begin{multicols}{2}
			\begin{enumerate}
				\item Spherical
				\item Conical
				\item Planar
				\item Cylindrical
			\end{enumerate}
		\end{multicols}

	\item The ingredient to be added to produce Aerated Cement Concrete, is
		\begin{multicols}{2}
			\begin{enumerate}
				\item Aluminium
				\item Calcium chloride
				\item Gypsum
				\item Sulphur
			\end{enumerate}
		\end{multicols}
		\newpage
	\item The cause of short column effect, during seismic occurrence, is due to
		\begin{multicols}{2}
			\begin{enumerate}
				\item Centralized rupture of the column
				\item Tearing of reinforcement bars
				\item Buckling of column
				\item Stress concentration
			\end{enumerate}
		\end{multicols}

	\item The solar protection system consisting of fixed slats or grids, outside a building façade in front of openings, is known as
		\begin{multicols}{2}
			\begin{enumerate}
				\item Brise-soleil
				\item Solarium
				\item Malqaf
				\item Trombe wall
			\end{enumerate}
		\end{multicols}

	\item The Indian property inscribed by UNESCO on the World Heritage List in the year 2018 is
		\begin{enumerate}
			\item Mattancherry Palace, Ernakulam
			\item The Victorian Gothic and Art Deco Ensembles of Mumbai
			\item Ancient Buddhist Site, Sarnath
			\item Mughal Gardens in Kashmir
		\end{enumerate}

	\item Typical features of Buddhist architecture are
		\begin{multicols}{2}
			\begin{enumerate}
				\item Mandapa, Chattri, Amalaka, Torana
				\item Stambha, Torana, Vimana, Harmika
				\item Vedika, Chattri, Torana, Harmika
				\item Vedika, Stupa, Chastity, Vimana
			\end{enumerate}
		\end{multicols}

	\item Identify the Queen closure
		\begin{multicols}{2}
			\begin{enumerate}
				\item \includegraphics[width=0.5\linewidth]{ar_q18_a.jpg}
				\item \includegraphics[width=0.5\linewidth]{ar_q18_b.jpg}
				\item \includegraphics[width=0.5\linewidth]{ar_q18_c.jpg}
				\item \includegraphics[width=0.5\linewidth]{ar_q18_d.jpg}
			\end{enumerate}
		\end{multicols}
		\newpage
	\item Identify the role of Vermiculate in vertical landscapes
		\begin{multicols}{2}
			\begin{enumerate}
				\item Fertilizer
				\item Holding material
				\item Binding material
				\item Water retention element
			\end{enumerate}
		\end{multicols}

	\item Which of the following parameters is essential to estimate the Envelope Performance Factor (EPF) of a building as per the Energy Conservation Building Code (ECBC), 2011?
		\begin{multicols}{2}
			\begin{enumerate}
				\item Building Type
				\item Maximum humidity
				\item Maximum and minimum monthly temperature
				\item Building occupancy duration
			\end{enumerate}
		\end{multicols}

	\item The illumination level of a room is 300 lux and the efficacy of the lamps is 60. The Light Power Density (LPD) of the room in Watt/m$^2$ is \underline{\hspace{2cm}}.

	\item The load on a RCC column is 150 kN. The soil bearing capacity is 80 kN/m$^2$. Assuming a factor of safety of 1.2, the side of the square column footing is \underline{\hspace{2cm}} meter \textit{(rounded off to one decimal place)}.

	\item A room is separated by a partition wall. The average intensities of sound in the source and receiving sides across the partition are $10^4$ W/m$^2$ and $10^1$ W/m$^2$ respectively. The transmission loss (TL) of the partition wall is \underline{\hspace{2cm}} dB.

	\item If the purchase price of 2BHK flat rises by 10 percent, the demand for such flats is observed to decrease by 8 percent. The price elasticity of the housing demand for 2BHK flats is \underline{\hspace{2cm}} \textit{(rounded off to one decimal place)}.

	\item ‘Threshold of enclosure’ created by vertical surfaces or series of vertical elements in an urban plaza, represented by the ratio of height and distance, is given by an angle of \underline{\hspace{2cm}} degrees \textit{(rounded off to one decimal place)}.
\end{enumerate}
\textbf{Q.26 -Q.55 carry two marks each}\hfill
\begin{enumerate}[label=Q.\arabic*,start=26]	
	\item Match the instruments in Column - I with the various types of surveying in Column - II and select the appropriate option.

		\begin{center}
			\begin{tabular}{|c|>{\raggedright\arraybackslash}p{4cm}|c|>{\raggedright\arraybackslash}p{6cm}|}
				\hline
				\multicolumn{2}{|c|}{\textbf{Column - I}} & \multicolumn{2}{c|}{\textbf{Column - II}} \\
				\hline
				P & Cross staff & 1 & Indoor wall to wall measurement \\
				\hline
				Q & Alidade & 2 & Traversing \\
				\hline
				R & Sextant & 3 & Chain survey \\
				\hline
				S & Distomat & 4 & Plane table survey \\
				\hline
				&           & 5 & Contour survey \\
				\hline
			\end{tabular}
		\end{center}

		\begin{multicols}{2}
			\begin{enumerate}
				\item P-3, Q-4, R-2, S-5
				\item P-2, Q-4, R-1, S-5
				\item P-5, Q-3, R-2, S-1
				\item P-3, Q-4, R-2, S-1
			\end{enumerate}
		\end{multicols}

	\item Match the characteristics of settlement systems in Column - I with their corresponding theory/rules in Column - II and select the appropriate option.

		\begin{center}
			\begin{tabular}{|c|>{\raggedright\arraybackslash}p{4cm}|c|>{\raggedright\arraybackslash}p{6cm}|}
				\hline
				\multicolumn{2}{|c|}{\textbf{Column - I}} & \multicolumn{2}{c|}{\textbf{Column - II}} \\
				\hline
				P & Primacy of settlements & 1 & Central place theory \\
				\hline
				Q & Settlement size and location & 2 & Gravity model \\
				\hline
				R & Random component in location of settlements & 3 & Rank size rule \\
				\hline
				S & Interaction between settlements & 4 & Entropy of settlements \\
				\hline
				&                                 & 5 & Core periphery model \\
				\hline
			\end{tabular}
		\end{center}

		\begin{multicols}{2}
			\begin{enumerate}
				\item P-4, Q-1, R-2, S-5
				\item P-2, Q-5, R-3, S-1
				\item P-3, Q-5, R-4, S-2
				\item P-3, Q-1, R-4, S-2
			\end{enumerate}
		\end{multicols}
	\item Match the architectural projects in Column - I with the architect in Column - II, and select the appropriate option.

		\begin{center}
			\begin{tabular}{|c|>{\raggedright\arraybackslash}p{4cm}|c|>{\raggedright\arraybackslash}p{6cm}|}
				\hline
				\multicolumn{2}{|c|}{\textbf{Column - I}} & \multicolumn{2}{c|}{\textbf{Column - II}} \\
				\hline
				P & India Habitat Centre, New Delhi & 1 & Christopher Charles Benninger \\
				\hline
				Q & United World Colleges (UWC), Mahindra College, Pune & 2 & Charles Correa \\ 
				\hline
				R & Brain \\& Cognitive Science Centre – MIT, Cambridge & 3 & Joseph Allen Stein \\
				\hline
				S & Habitat 67, Montreal & 4 & Norman Foster \\
				\hline
				& & 5 & Moshe Safdi \\
				\hline
			\end{tabular}
		\end{center}
		\begin{multicols}{2}
			\begin{enumerate}
				\item P-3, Q-1, R-2, S-5
				\item P-1, Q-2, R-5, S-3
				\item P-2, Q-1, R-5, S-4
				\item P-3, Q-4, R-1, S-5
			\end{enumerate}
		\end{multicols}
		\newpage
	\item Match the Name of the book provided in Column - I with the corresponding author in Column - II and select the appropriate option.
		\begin{center}
			\begin{tabular}{|c|>{\raggedright\arraybackslash}p{5.8cm}|c|>{\raggedright\arraybackslash}p{5.8cm}|}
				\hline
				\multicolumn{2}{|c|}{\textbf{Column - I}} & \multicolumn{2}{c|}{\textbf{Column - II}} \\
				\hline
				P & Earthscape & 1 & Ian McHarg \\ 
				\hline
				Q & Synthesis of Form & 2 & John O Simonds \\
				\hline
				R & Design with Nature & 3 & Christopher Alexander \\
				\hline
				S & The City of Tomorrow and its Planning & 4 & Lewis Mumford \\
				\hline
				&  & 5 & Le Corbusier \\
				\hline
			\end{tabular}
		\end{center}
		\begin{multicols}{2}
			\begin{enumerate}
				\item P-2, Q-3, R-1, S-5
				\item P-5, Q-2, R-3, S-4
				\item P-2, Q-5, R-1, S-4
				\item P-2, Q-1, R-4, S-5
			\end{enumerate}
		\end{multicols}
	\item Match the thermal properties in the Column - I and their respective units in Column - II and select the appropriate option.


		\begin{center}
			\begin{tabular}{|c|>{\raggedright\arraybackslash}p{5.8cm}|c|>{\raggedright\arraybackslash}p{5.8cm}|}
				\hline
				\multicolumn{2}{|c|}{\textbf{Column - I}} & \multicolumn{2}{c|}{\textbf{Column - II}} \\
				\hline
				P & Thermal Resistance & 1 & J kg$^{-1}$ °C$^{-1}$ \\
				\hline
				Q & Thermal Transmittance & 2 & W m$^{-1}$ °C$^{-1}$ \\
				\hline
				R & Specific Heat & 3 & W m$^{-2}$ °C$^{-1}$ \\
				\hline
				S & Thermal Conductivity & 4 & m$^{2}$ °C W$^{-1}$ \\
				\hline
				&  & 5 & J m$^{3}$ °C$^{-1}$ \\
				\hline
			\end{tabular}
		\end{center}

		\begin{multicols}{2}
			\begin{enumerate}
				\item P-4, Q-1, R-5, S-2
				\item P-4, Q-3, R-1, S-2
				\item P-5, Q-3, R-1, S-4
				\item P-3, Q-4, R-2, S-1
			\end{enumerate}
		\end{multicols}


	\item Match the application in the field of construction in the Column - I and the respective items in Column - II and select the appropriate option.

		\begin{center}
			\begin{tabular}{|c|>{\raggedright\arraybackslash}p{5cm}|c|>{\raggedright\arraybackslash}p{6cm}|}
				\hline
				\multicolumn{2}{|c|}{\textbf{Column - I}} & \multicolumn{2}{c|}{\textbf{Column - II}} \\
				\hline
				P i& Polytetrafluoroethylene (PTFE) membrane & 1 & Tendon \\
				\hline
				Q & Isolated compression component inside a network of continuous tensile member & 2 & TMT \\
				\hline
				R & Cable used for pre-stressed concrete & 3 & Tensegrity \\
				\hline
				S & Reinforcement bar used in RCC construction & 4 & TMD \\
				\hline
				& & 5 & Teflon \\
				\hline
			\end{tabular}
		\end{center}

		\begin{multicols}{2}
			\begin{enumerate}
				\item P-5, Q-1, R-4, S-3
				\item P-4, Q-3, R-1, S-5
				\item P-5, Q-3, R-1, S-2
				\item P-3, Q-4, R-2, S-1
			\end{enumerate}
		\end{multicols}


	\item Match the following types of masonry joints in Column - I with their corresponding description in Column - II, and select the appropriate option.


		\begin{center}
			\begin{tabular}{|c|>{\raggedright\arraybackslash}p{4cm}|c|>{\raggedright\arraybackslash}p{6cm}|}
				\hline
				\multicolumn{2}{|c|}{\textbf{Column - I}} & \multicolumn{2}{c|}{\textbf{Column - II}} \\
				\hline
				P & \includegraphics[height=2.5cm,width=\linewidth]{ar_q32_a.jpg} & 1 & Struck \\
				\hline
				Q &  \includegraphics[height=2.5cm,width=\linewidth]{ar_q32_b.jpg} & 2 & Weathered \\
				\hline
				R &  \includegraphics[height=2.5cm,width=\linewidth]{ar_q32_c.jpg} & 3 & Raked \\
				\hline
				S &  \includegraphics[height=2.5cm,width=\linewidth]{ar_q32_d.jpg} & 4 & Beaded \\
				\hline
				&          & 5 & Concave \\
				\hline
			\end{tabular}
		\end{center}

		\begin{multicols}{2}
			\begin{enumerate}
				\item P-1, Q-3, R-2, S-4
				\item P-4, Q-3, R-2, S-5
				\item P-3, Q-4, R-5, S-2
				\item P-4, Q-3, R-1, S-5
			\end{enumerate}
		\end{multicols}

	\item Match the following in Column - I with their suitable description in Column - II, and select the appropriate option.


		\begin{center}
			\begin{tabular}{|c|>{\raggedright\arraybackslash}p{4cm}|c|>{\raggedright\arraybackslash}p{6cm}|}
				\hline
				\multicolumn{2}{|c|}{\textbf{Column - I}} & \multicolumn{2}{c|}{\textbf{Column - II}} \\
				\hline
				P & Tolerance & 1 & 100 mm \\
				\hline
				Q & Precast concrete rings for wells & 2 & Non modular dimension \\
				\hline
				R & M & 3 & Acceptable variation \\
				\hline
				S & Weather joints & 4 & 3D-prefabricate \\
				\hline
				& & 5 & Resilient sealants \\
				\hline
			\end{tabular}
		\end{center}

		\begin{multicols}{2}
			\begin{enumerate}
				\item P-2, Q-4, R-1, S-3
				\item P-4, Q-3, R-3, S-5
				\item P-1, Q-2, R-3, S-4
				\item P-3, Q-4, R-3, S-5
			\end{enumerate}
		\end{multicols}


	\item Match the units provided in Column - I with their corresponding items in Column - II, and select the appropriate option.

		\begin{center}
			\begin{tabular}{|c|>{\raggedright\arraybackslash}p{4cm}|c|>{\raggedright\arraybackslash}p{6cm}|}
				\hline
				\multicolumn{2}{|c|}{\textbf{Column - I}} & \multicolumn{2}{c|}{\textbf{Column - II}} \\
				\hline
				P & dB & 1 & Sound Intensity \\
				\hline
				Q & Phon & 2 & Absorption of sound \\
				\hline
				R & W/m$^2$ & 3 & Frequency of sound \\
				\hline
				S & Sabine & 4 & Loudness \\
				\hline
				& & 5 & Sound pressure level \\
				\hline
			\end{tabular}
		\end{center}

		\begin{multicols}{2}
			\begin{enumerate}
				\item P-5, Q-1, R-4, S-3
				\item P-2, Q-3, R-4, S-5
				\item P-1, Q-2, R-3, S-4
				\item P-5, Q-4, R-1, S-2
			\end{enumerate}
		\end{multicols}

	\item Match the scientific names of the trees provided in Column - I with the corresponding color of their bloom in Column - II, and select the appropriate option.

		\begin{center}
			\begin{tabular}{|c|>{\raggedright\arraybackslash}p{5.8cm}|c|>{\raggedright\arraybackslash}p{5.8cm}|}
				\hline
				\multicolumn{2}{|c|}{\textbf{Column - I}} & \multicolumn{2}{c|}{\textbf{Column - II}} \\
				\hline
				P & \textit{Cassia fistula} & 1 & White \\
				\hline
				Q & \textit{Lagerstroemia flos-reginae} & 2 & Red \\
				\hline
				R & \textit{Cordia sebestena} & 3 & Blue \\
				\hline
				S & \textit{Plumeria alba} & 4 & Yellow \\
				\hline
				&  & 5 & Mauve \\
				\hline
			\end{tabular}
		\end{center}

		\begin{multicols}{2}
			\begin{enumerate}
				\item P-4, Q-5, R-4, S-1
				\item P-1, Q-5, R-2, S-3
				\item P-5, Q-4, R-1, S-3
				\item P-4, Q-5, R-2, S-1
			\end{enumerate}
		\end{multicols}

	\item Match the items in Column - I and their respective location in building/site in Column - II, and select the appropriate option.

		\begin{center}
			\begin{tabular}{|c|>{\raggedright\arraybackslash}p{5.8cm}|c|>{\raggedright\arraybackslash}p{5.8cm}|}
				\hline
				\multicolumn{2}{|c|}{\textbf{Column - I}} & \multicolumn{2}{c|}{\textbf{Column - II}} \\
				\hline
				P & Nahani Trap & 1 & Between waste water pipe and main house drain \\
				\hline
				Q & Gully Trap & 2 & Between septic tank and soak pit \\
				\hline
				R & Bottle Trap & 3 & Junction of house drain and sewer \\
				\hline
				S & Intercepting Trap & 4 & Bathroom and kitchen floor \\
				\hline
				&  & 5 & Below the wash basin \\
				\hline
			\end{tabular}
		\end{center}

		\begin{multicols}{2}
			\begin{enumerate}
				\item P-4, Q-5, R-2, S-3
				\item P-5, Q-1, R-3, S-2
				\item P-1, Q-2, R-3, S-4
				\item P-3, Q-4, R-5, S-2
			\end{enumerate}
		\end{multicols}

	\item As per the Handbook on Barrier Free and Accessibility, CPWD - 2014, match the design guidelines in Column - I with their appropriate standards in Column - II and select the appropriate option.


		\begin{center}
			\begin{tabular}{|c|>{\raggedright\arraybackslash}p{6cm}|c|>{\raggedright\arraybackslash}p{3.5cm}|}
				\hline
				\multicolumn{2}{|c|}{\textbf{Column - I}} & \multicolumn{2}{c|}{\textbf{Column - II}} \\
				\hline
				P & Minimum clear width of ramp & 1 & 600 mm \\
				\hline
				Q & Maximum height of wash basin (rim) above finished floor level & 2 & 1500 mm \\
				\hline
				R & Minimum length of grab rail & 3 & 750 mm \\
				\hline
				S & Minimum clear width for maneuvering space (wheelchair) & 4 & 900 mm \\
				\hline
				& & 5 & 1800 mm \\
				\hline
			\end{tabular}
		\end{center}

		\begin{multicols}{2}
			\begin{enumerate}
				\item P-3, Q-4, R-1, S-5
				\item P-5, Q-3, R-2, S-4
				\item P-5, Q-3, R-1, S-2
				\item P-1, Q-4, R-3, S-1
			\end{enumerate}
		\end{multicols}


	\item Match the contemporary Urban Design Movements listed in Column - I with the corresponding principles listed in Column - II and select the appropriate option.
		\begin{center}
			\begin{tabular}{|c|>{\raggedright\arraybackslash}p{5cm}|c|>{\raggedright\arraybackslash}p{6cm}|}
				\hline
				\multicolumn{2}{|c|}{\textbf{Column - I}} & \multicolumn{2}{c|}{\textbf{Column - II}} \\
				\hline
				P & Park Movement & 1 & Self-contained, self-sufficient community surrounded by green belts \\
				\hline
				Q & New Urbanism & 2 & Revival of the relationship between man and nature \\
				\hline
				R & City Beautiful Movement & 3 & Relationship between work and living, environmental sustainability \\
				\hline
				S & Garden City and New Town Movement & 4 & Unity, cohesion and balanced relationship between urban components and elements \\
				\hline
				& & 5 & Technical and socio economic processes resulting in growth, energy production and waste elimination \\
				\hline
			\end{tabular}
		\end{center}
		\begin{multicols}{2}
			\begin{enumerate}
				\item P-2, Q-3, R-4, S-1
				\item P-1, Q-5, R-3, S-2
				\item P-5, Q-3, R-1, S-2
				\item P-2, Q-5, R-4, S-1
			\end{enumerate}
		\end{multicols}


	\item Match the figures of vaults in Column - I with their corresponding types in Column - II and select the appropriate option.

		\begin{center}
			\begin{tabular}{|c|>{\raggedright\arraybackslash}p{4cm}|c|>{\raggedright\arraybackslash}p{6cm}|}
				\hline
				\multicolumn{2}{|c|}{\textbf{Column - I}} & \multicolumn{2}{c|}{\textbf{Column - II}} \\
				\hline
				P &  \includegraphics[height=3cm,width=\linewidth]{ar_q39_a.jpg} & 1 & Ribbed \\
				\hline
				Q &  \includegraphics[height=3cm,width=\linewidth]{ar_q39_b.jpg}& 2 & Fan \\
				\hline
				R &  \includegraphics[height=3cm,width=\linewidth]{ar_q39_c.jpg} & 3 & Barrel \\
				\hline
				S &  \includegraphics[height=3cm,width=\linewidth]{ar_q39_d.jpg}& 4 & Groin \\
				\hline
				& & 5 & Nubian \\
				\hline
			\end{tabular}
		\end{center}

		\begin{multicols}{2}
			\begin{enumerate}
				\item P-3, Q-4, R-1, S-2
				\item P-3, Q-1, R-4, S-5
				\item P-2, Q-1, R-5, S-3
				\item P-2, Q-3, R-1, S-5
			\end{enumerate}
		\end{multicols}

	\item A colony of 50 people is served by a septic tank. The rate of water supply is 90 lpd in the colony and 40\% of it is going to the septic tank. The retention period of the tank is 24 hours. The length of the septic tank is \underline{\hspace{2cm}} meter \textit{(rounded off to two decimal places)}.\\
		Assume, storage capacity/person = 0.085m$^3$ (3 years)\\
		Space for digestion = 0.0425 m$^3$/person\\
		Depth of tank = 1.4 m\\
		Length: Width = 2:1

	\item A cone, with a base of 10 cm diameter and axis of 12 cm, is lying on Horizontal Plane (HP) along its generator. The internal angle which the base of the cone makes with the HP is \underline{\hspace{2cm}} degrees.

	\item A public utility building of 5000 m$^2$ was constructed 5 years before, on a site of 1 hectare. The present value of open land in that location is Rs. 100/m$^2$ and present construction cost of such building is Rs. 2500/m$^2$. If the value of the building is assumed to be depreciating at a constant rate of 6 percent per annum, then the present value of the property using “Valuation by Cost Method” is \underline{\hspace{2cm}} (in Rs. lakhs) \textit{(rounded off to one decimal place)}.

	\item A residential area of 20 hectares is planned for three different types of plots of 500 m$^2$, 300 m$^2$ and 200 m$^2$ with numbers of plot in each category are 100, 120 and 150 respectively. The rest of the area is allocated for roads and facilities such as schools, shops and parks. Each plot has one dwelling unit and the average household size is 5 persons.\\
		The net residential density of the area in persons per hectare is \underline{\hspace{2cm}}.

	\item In a single lane road, traffic volume of 1000 vehicle/h moving at 20 km/h, comes to a halt due to an accident. If jam density is 150 vehicle/km, the velocity of the shock wave generated (in absolute value) is \underline{\hspace{2cm}} km/h.

	\item In a site map, a rectangular residential plot measures 150 mm $\times$ 40 mm, and the width of the front road in the map measures 16 mm. Actual width of the road is 4 m. If the permissible F.A.R. is 1.2, the maximum built-up area for the residential building will be \underline{\hspace{2cm}} m$^2$.

	\item The internal dimension of a room is 10m $\times$ 10m $\times$ 4m (height). The total area of the doors and windows are 16 m$^2$. Keeping the doors and windows closed, the reverberation time of the room becomes 1.2 second. Assume all the interior surfaces including doors and windows have same sound absorption coefficient. If all the doors and windows of the room are kept fully open, the reverberation time will be \underline{\hspace{2cm}} second \textit{(rounded off to two decimal places)}.

	\item A depressed portion of a land is identified by three closed contours, as shown in the figure below. The area bounded by the contour lines are 6 m$^2$, 24 m$^2$ and 96 m$^2$ respectively.\\
		\begin{center}{
				\includegraphics[width=0.5\linewidth]{ar_q47.jpg}
		}\end{center}
		The contour interval is 1 m. Using prismoidal method, the volume of the earth needed to fill the land depression is \underline{\hspace{2cm}} m$^3$.

	\item Solar panels are proposed to be installed on a building roof top to generate electricity. The size of each solar panel is 2 m$^2$. The efficiency of each panel is 75\%. The orientations of the solar panel and related solar data are given in the table below.\hfill


		As per the above proposal \underline{\hspace{2cm}} kWh solar power will be generated daily. \textit{(rounded off to one decimal place)}

	\item A power shovel is having 1.8 m$^3$ excavation output per batch of operation. The average cycle time of the batch operation is 45 seconds. The lost time per hour of the excavation activity is 10 minutes. Assume six working hours of operation per day. The amount of soil excavated by the power shovel in 1 day of operation will be \underline{\hspace{2cm}} m$^3$. \textit{(rounded off to two decimal places)}.


	\item A room having dimension 12 m $\times$ 10 m $\times$ 3.5 m is required to be mechanically ventilated by air-conditioning. The temperature difference between outdoor ambient air and the supply air is 12 $^\circ$C. Consider three air exchanges per hour. The volumetric specific heat of the air is 1250 J/m$^3$ $^\circ$C. Assume one ton of refrigeration (TR) is equal to 3.5 kW. The capacity of the air-conditioner for the room in TR will be \underline{\hspace{2cm}}.

	\item A simply supported beam AB has a clear span of 7 meter. The bending moment diagram (BMD) of the beam due to a single concentrated load is shown in the figure below.

		\begin{center}{
				\includegraphics[width=0.5\linewidth]{ar_q51.jpg}
		}\end{center}

	\item For a symmetrical trapezoidal open drain in a landscape with grass and loose rock surface, the velocity of flow of water is \underline{\hspace{2cm}} m/sec, \textit{(rounded off to two decimal places)}, given the following data. \\
		Water edge width at the top = 750 mm \\
		Water edge width at the bottom = 450 mm \\
		Water depth = 600 mm \\
		Manning’s coefficient of roughness = 0.05 \\
		Slope along the drain = 1 in 250

	\item The stack pressure is created by 10 m height of stack and 15°C temperature difference. The motive force due to the stack pressure over a cross section area of 2.5 m² is \underline{\hspace{2cm}} N.

	\item An industrial building contains 3000 kg of combustible materials, in dry state, distributed over three rooms of area 100 m², 500 m² and 300 m² each, in a proportion of 30\%, 50\% and 20\% of the contents, respectively. Calorific value of the material is 4400 Kcal/kg. The Total Fire Load of the rooms is equal to \underline{\hspace{2cm}} Kcal/m².

	\item A simple truss is shown in the figure below. The truss is loaded with horizontal and vertical force 15 kN and 25 kN, respectively. The force in the member AB will be \underline{\hspace{2cm}} kN.
		\begin{center}{
				\includegraphics[width=0.5\linewidth]{ar_q55.jpg}
		}\end{center}

		\begin{center}
			\textbf{END OF QUESTION  PAPER}
		\end{center}



\end{enumerate}
\end{document}

